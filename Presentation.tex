\documentclass{beamer}
\usepackage[utf8]{inputenc}
\usepackage{algorithm,algorithmic} 
\usepackage{amsmath}
\usepackage{ulem}
\usepackage{ amssymb }
\usepackage{mathtools}
\usepackage{amsthm}
\usepackage{ stmaryrd }
\usepackage{ dsfont }
\theoremstyle{definition}
\newtheorem{mydef}{Definition}
\newtheorem{mylem}{Lemma}
\newtheorem{mythe}{Theorem}
\newtheorem{mycol}{Corollary}
\usepackage{color}
\usepackage{xcolor}
\usepackage{paralist}
\usepackage{tikz}
\usetikzlibrary{positioning}
\usetikzlibrary{scopes}
\usepackage{kbordermatrix}
\usetikzlibrary{shapes,snakes,backgrounds}
\usepackage[absolute,overlay]{textpos}
\usetheme{Warsaw}  %% Themenwahl
\beamertemplatenavigationsymbolsempty
\newcommand{\xdownarrow}[1]{%
  {\left\downarrow\vbox to #1{}\right.\kern-\nulldelimiterspace}
}
\tikzset{cloud/.pic={
\node[cloud, cloud puffs=8.5,cloud puff arc=90, aspect=2, draw, text width=2.5cm
    ] () at (0,0) {\tikzpictext};
}}
\def\firstcircle{(0,0) circle (1.5cm)}
\def\secondcircle{(0:2cm) circle (1.5cm)}
\makeatletter
\newenvironment<>{proofs}[1][\proofname]{%
    \par
    \def\insertproofname{#1\@addpunct{.}}%
    \usebeamertemplate{proof begin}#2}
  {\usebeamertemplate{proof end}}
\makeatother
\setbeamertemplate{footline}[frame number]
\title{A Tableau algorithm for $\mathcal{ALCSCC}$}
\author{Ryny Khy}
\date{January 12, 2021}
\begin{document}
\maketitle
\frame{\tableofcontents}
\section{Introduction}
\begin{frame}\frametitle{Motivation}
\begin{itemize}
\item Reasoning in data base
\pause
\item Satisfiability check for reasoning
\pause
\item Tableau algorithm for satisfiability check
\end{itemize}
\end{frame}
\begin{frame}\frametitle{Tableau Algorithm}
Main Idea (for $\mathcal{ALCQ}$):
\begin{align*}
x:A\sqcap B\,\sqcap \geq\, 2\,r.C
\end{align*}
\vspace*{-1cm}
\begin{minipage}[t]{0.6\textwidth}
\centering
\only<1,2>{
\begin{tikzpicture}
\pic (c1) at (0,0) [pic text=$x$ must be in $A$]{cloud};
\end{tikzpicture}
}
\only<3,4>{
\begin{tikzpicture}
\pic (c1) at (0,0) [pic text=$x$ must be in $B$]{cloud};
\end{tikzpicture}
}
\only<5,6,7>{
\begin{tikzpicture}
\pic (c1) at (0,0) [pic text=$x$ has at least two $r$-successor in $C$]{cloud};
\end{tikzpicture}
}
\end{minipage}%
\begin{minipage}[t]{0.4\textwidth}
\centering
\begin{tikzpicture}
\node[draw=black,ellipse] (x) at (5,1.5) {$x$};
\only<2,3>{
\node[right = 0.5cm of x] {$A$\hphantom{$, B$}};
}
\only<4,5,6>{
\node[right = 0.5cm of x] {$A,B$};
}
\pause
\pause
\pause
\pause
\pause
\node[draw=black,ellipse] (y1) at (4,-1.25){$y_1$};
\node [right = 0.5cm of y1] {$C$};
\draw[->] (x) edge node[left] {$r$} (y1);
\pause
\node[draw=black,ellipse] (y2) at (6,-1.25){$y_2$};
\node [right = 0.5cm of y2] {$C$};
\draw[->] (x) edge node[left] {$r$} (y2);
\end{tikzpicture}
\end{minipage}
\end{frame}
\begin{frame}\frametitle{Goal}
\centering
Tableau algorithm for $\mathcal{ALCSCC}$ concepts
\end{frame}
\begin{frame}\frametitle{$\mathcal{ALCSCC}$: successors}
\begin{center}
\begin{tikzpicture}
\node[draw=black,cross out] at (0,0) {$\exists r.C$};
\node[draw=black,cross out] at (2.5,0) {$\forall r.C$};
\node[draw=black,cross out] at (5.25,0) {$\leq\,m\, r.C$};
\node[draw=black,cross out] at (8.25,0) {$\geq\,m\, r.C$};
\node[draw=black] at (4,-1.75){$succ(c)$};
\end{tikzpicture}
\end{center}
$c$: \textbf{set constraint} or \textbf{cardinality constraint}
\end{frame}
\begin{frame}\frametitle{$\mathcal{ALCSCC}$: constraints}
\begin{minipage}[t]{0.5\textwidth}
\raggedright
\textbf{set constraint:}
\begin{itemize}
\item $isChild\cap Female$\\ $\subseteq Daughter$
\end{itemize}
\end{minipage}%
\begin{minipage}[t]{0.5\textwidth}
\raggedright
\textbf{cardinality constraint}
\begin{itemize}
\item $2\,dvd\,|hasLegs|$
\item $|Edges|\leq |Nodes|$
\end{itemize}
\end{minipage}
\end{frame}
\begin{frame}\frametitle{$\mathcal{ALCSCC}$: constraints}
\begin{minipage}[t]{0.5\textwidth}
\raggedright
\textbf{set constraint:}
\begin{itemize}
\item $succ(2\,dvd\,|Edges|)\subseteq$\\
		$ succ(|Edges|=|Nodes|)$
\end{itemize}
\centering
\begin{tikzpicture}[transform canvas={scale=.75}]
\node (x1) at (0,0) {$x_1$};
\node (x2) at (0,-1) {$x_2$};
\node (e1) at (-1.5,-2) {$e_1$};
\node (e2) at (-0.5,-2) {$e_2$};
\node (n1) at (0.5,-2) {$n_1$};
\node (n2) at (1.5,-2) {$n_2$};
\draw[->] (x1) -- (x2);
\draw[->] (x2) -- (e1);
\draw[->] (x2) -- (e2);
\draw[->] (x2) -- (n1);
\draw[->] (x2) -- (n2);
\end{tikzpicture}
\end{minipage}%
\begin{minipage}[t]{0.5\textwidth}
\raggedright
\textbf{cardinality constraint}
\begin{itemize}
\item $succ(2\,dvd\,|Edges|)\leq$\\
		$ succ(|Edges|=|Nodes|)$
\end{itemize}
\centering
\begin{tikzpicture}[transform canvas={scale=.75}]
\node (x1) at (0,0) {$x_1$};
\node (x2) at (-0.75,-1) {$x_2$};
\node (x3) at (0.75,-1) {$x_3$};
\node (e1) at (-1.5,-2) {$e_1$};
\node (e2) at (-0.5,-2) {$e_2$};
\node (e3) at (0.5,-2) {$e_3$};
\node (n1) at (1.5,-2) {$n_1$};
\draw[->] (x1) -- (x2);
\draw[->] (x1) -- (x3);
\draw[->] (x2) -- (e1);
\draw[->] (x2) -- (e2);
\draw[->] (x3) -- (e3);
\draw[->] (x3) -- (n1);
\end{tikzpicture}
\end{minipage}
\end{frame}
\section{Difficulties with $\mathcal{ALCSCC}$}
\begin{frame}\frametitle{Problem with successors constraints}
\begin{align*}
x:\, succ(|s|>1)\,\sqcap\, succ(|r|\leq|s|)\, \sqcap\,succ(|r|>|s|)
\end{align*}
\pause
\begin{itemize}
\item endless loop of adding $r$- and $s$-successors
\pause
\item blocking?
\end{itemize}
\end{frame}
\begin{frame}\frametitle{Problem with blocking}
\only<1>{
\begin{align*}
x:succ(3\cdot |r|\leq 5\cdot|s|)\,\sqcap\,succ(5\cdot|s|\leq 3\cdot|r|)\,\sqcap\,succ(|r|>1)
\end{align*}
\begin{center}
\begin{tikzpicture}
\node[draw=black,ellipse] (s) at (0,0) {0};
\node[draw=black,ellipse] (r) at (4,0) {0};
\node[above = 0.3cm of s] {$s$-successors};
\node[above = 0.3cm of r] {$r$-successors};
\end{tikzpicture}
\end{center}
}
\only<2>{
\begin{align*}
x:succ(3\cdot |r|\leq 5\cdot|s|)\,\sqcap\,succ(5\cdot|s|\leq 3\cdot|r|)\,\sqcap\,\underline{succ(|r|>1)}
\end{align*}
\begin{center}
\begin{tikzpicture}
\node[draw=black,ellipse] (s) at (0,0) {0};
\node[draw=black,ellipse] (r) at (4,0) {1};
\node[above = 0.3cm of s] {$s$-successors};
\node[above = 0.3cm of r] {$r$-successors};
\end{tikzpicture}
\end{center}
}
\only<3>{
\begin{align*}
x:\underline{succ(3\cdot |r|\leq 5\cdot|s|)}\,\sqcap\,succ(5\cdot|s|\leq 3\cdot|r|)\,\sqcap\,succ(|r|>1)
\end{align*}
\begin{center}
\begin{tikzpicture}
\node[draw=black,ellipse] (s) at (0,0) {0};
\node[draw=black,ellipse] (r) at (4,0) {1};
\node[above = 0.3cm of s] {$s$-successors};
\node[above = 0.3cm of r] {$r$-successors};
\end{tikzpicture}
\end{center}
}
\only<4>{
\begin{align*}
x:\underline{succ(3\cdot |r|\leq 5\cdot|s|)}\,\sqcap\,succ(5\cdot|s|\leq 3\cdot|r|)\,\sqcap\,succ(|r|>1)
\end{align*}
\begin{center}
\begin{tikzpicture}
\node[draw=black,ellipse] (s) at (0,0) {1};
\node[draw=black,ellipse] (r) at (4,0) {1};
\node[above = 0.3cm of s] {$s$-successors};
\node[above = 0.3cm of r] {$r$-successors};
\end{tikzpicture}
\end{center}
}
\only<5>{
\begin{align*}
x:succ(3\cdot |r|\leq 5\cdot|s|)\,\sqcap\,\underline{succ(5\cdot|s|\leq 3\cdot|r|)}\,\sqcap\,succ(|r|>1)
\end{align*}
\begin{center}
\begin{tikzpicture}
\node[draw=black,ellipse] (s) at (0,0) {1};
\node[draw=black,ellipse] (r) at (4,0) {1};
\node[above = 0.3cm of s] {$s$-successors};
\node[above = 0.3cm of r] {$r$-successors};
\end{tikzpicture}
\end{center}
}
\only<6>{
\begin{align*}
x:succ(3\cdot |r|\leq 5\cdot|s|)\,\sqcap\,\underline{succ(5\cdot|s|\leq 3\cdot|r|)}\,\sqcap\,succ(|r|>1)
\end{align*}
\begin{center}
\begin{tikzpicture}
\node[draw=black,ellipse] (s) at (0,0) {1};
\node[draw=black,ellipse] (r) at (4,0) {2};
\node[above = 0.3cm of s] {$s$-successors};
\node[above = 0.3cm of r] {$r$-successors};
\end{tikzpicture}
\end{center}
}
\only<7>{
\begin{align*}
x:\underline{succ(3\cdot |r|\leq 5\cdot|s|)}\,\sqcap\,succ(5\cdot|s|\leq 3\cdot|r|)\,\sqcap\,succ(|r|>1)
\end{align*}
\begin{center}
\begin{tikzpicture}
\node[draw=black,ellipse] (s) at (0,0) {1};
\node[draw=black,ellipse] (r) at (4,0) {1};
\node[above = 0.3cm of s] {$s$-successors};
\node[above = 0.3cm of r] {$r$-successors};
\end{tikzpicture}
\end{center}
}
\end{frame}
\begin{frame}\frametitle{QFBAPA formula and solver}
\begin{align*}
x:\textcolor{red}{succ(3\cdot |r|\leq 5\cdot|s|)}\,\sqcap\,\textcolor{blue}{succ(5\cdot|s|\leq 3\cdot|r|)}\,\sqcap\,\textcolor{orange}{succ(|r|>1)}
\end{align*}
\pause
\centering $\big\downarrow$
\begin{align*}
\textcolor{red}{3\cdot |X_r|\leq 5\cdot |X_s|}\,\wedge\, \textcolor{blue}{5\cdot |X_s|\leq 3\cdot |X_r|}\,\wedge\, \textcolor{orange}{|X_r|>1}
\end{align*}
\pause
\centering $\Big\downarrow$
\begin{align*}
X_r=\{r_1,r_2,\dots,r_5\}\text{ and }X_s=\{s_1,s_2,s_3\}
\end{align*}
\end{frame}
\section{Dealing with successor constraints}
\begin{frame}\frametitle{QFBAPA formula and solver}
\only<1>{
What about
\begin{align*}
\textcolor{red}{3\cdot |X_r|\leq 5\cdot |X_s|}\wedge \textcolor{blue}{5\cdot |X_s|\leq 3\cdot |X_r|}\wedge \textcolor{orange}{|X_r|>1}
\end{align*}
\centering $\big\downarrow$
\begin{align*}
|X_r|=10\text{ and }|X_s|=6
\end{align*}
}
\only<2>{
What about
\begin{align*}
\textcolor{red}{3\cdot |X_r|\leq 5\cdot |X_s|}\wedge \textcolor{blue}{5\cdot |X_s|\leq 3\cdot |X_r|}\wedge \textcolor{orange}{|X_r|>1}
\end{align*}
\centering $\Big\downarrow$
\begin{align*}
|X_r|=100\text{ and }|X_s|=60
\end{align*}
}
\end{frame}
\begin{frame}\frametitle{QFBAPA formula and solver}
\begin{center}
$\rightarrow$ Upper bound\\
\vspace*{1cm}
$\rightarrow$ ILP
\end{center}
\end{frame}
\begin{frame}\frametitle{Venn regions}
\begin{itemize}
\item[]\textbf{Problem}: Are the variables disjoint or not?\pause
\item[]\textbf{Solution}: Venn region
\end{itemize}
\centering
\begin{tikzpicture}
% left hand
\scope
\clip
(1.5,0) circle (1.5);
(0,0) circle (1.5);
\endscope
% right hand
\scope
\clip
(0,0) circle (1.5);
(1.5,0) circle (1.5);
\endscope
% outline
\draw (0,0) circle (1.5) (0,1.5)  node [text=black,above = 0.1cm] {$X_s$}
      (1.5,0) circle (1.5) (1.5,1.5)  node [text=black,above = 0.1cm] {$X_r$} ;
\end{tikzpicture}
\end{frame}
\begin{frame}{Venn regions}
Two variables $X_s$ and $X_r$ $\rightarrow$ four Venn regions:\\
\begin{minipage}[t]{0.5\textwidth}
\begin{align*}
&\textcolor{red}{v_1=X_s\cap X_r}\\
&\textcolor{green}{v_2=X_s^\neg\cap X_r}\\
&\textcolor{blue}{v_3=X_s\cap X_r^\neg}\\
&v_4=X_s^\neg\cap X_r^\neg
\end{align*}
\vspace*{0.75cm}
\end{minipage}%
\begin{minipage}[t]{0.5\textwidth}

\vspace*{1.75cm}
\hspace*{1.5cm}
\begin{tikzpicture}[transform canvas={scale=.75}]
\begin{scope}
      \clip \firstcircle;
      \fill[red] \secondcircle;
    \end{scope}
\begin{scope}
        \begin{scope}[even odd rule]% first circle without the second
            \clip \secondcircle (-3,-3) rectangle (3,3);
        \fill[green] \firstcircle;
        \end{scope}
    \end{scope}
    \begin{scope}
        \begin{scope}[even odd rule]% first circle without the second
            \clip \firstcircle (-3,-3) rectangle (4,3);
        \fill[blue] \secondcircle;
        \end{scope}
        \draw \firstcircle node {$X_r$};
        \draw \secondcircle node {$X_s$};
    \end{scope}
\end{tikzpicture}
\end{minipage}
\pause
$X_s=\textcolor{red}{v_1}\cup \textcolor{blue}{v_3}$ and $X_r=\textcolor{red}{v_1}\cup \textcolor{green}{v_2}$
\end{frame}
\begin{frame}{Venn regions}
Franz Baader \footnote{\tiny{A New Description Logic with Set Constraints and Cardinality
Constraints on Role Successors. Springer International Publishing: 43-59, 2017}}:\\
\vspace*{0.5cm}
For every QFBAPA formula $\phi$ there is a number $N$, which is polynomial in the size of $\phi$ and can be computed in polynomial time such, that for every solution $\sigma$ of $\phi$ there exists a solution $\sigma^\prime$ of $\phi$ such that:
\begin{itemize}
\item $|\{v|v\text{ is a Venn region and }\sigma^\prime(v)\neq \emptyset\}|\leq N$
\item $\{v|v\text{ is a Venn region and }\sigma^\prime(v)\neq \emptyset\}\subseteq \{v|v\text{ is a Venn region and }\sigma(v)\neq \emptyset\}$
\end{itemize}
\end{frame}
\begin{frame}{Venn regions}
In short:
\vspace*{0.5cm}
\begin{minipage}[t]{0.5\textwidth}
\begin{align*}
&2^n\text{ Venn regions,}\\
&n:\text{ amount of set variables}
\end{align*}
\centering
\vspace*{0.5cm}
\includegraphics[scale=0.3]{redcross}
\end{minipage}%
\begin{minipage}[t]{0.5\textwidth}
\begin{align*}
&N \text{ Venn regions, } \\
&N: \text{ polynomial in the size of formula}
\end{align*}
\centering
\includegraphics[scale=0.5]{greentick}
\end{minipage}%
\end{frame}
\begin{frame}\frametitle{Transforming formula into ILP}
\begin{align*}
\textcolor{red}{3\cdot |X_r|\leq 5\cdot |X_s|}\wedge \textcolor{blue}{5\cdot |X_s|\leq 3\cdot |X_r|}\wedge \textcolor{orange}{|X_r|>1}
\end{align*}
\centering $\big\downarrow$
\begin{align*}
\textcolor{red}{-5\cdot |X_s|+3\cdot |X_r|\leq 0}\\
\textcolor{blue}{5\cdot |X_s|-3\cdot |X_r|\leq 0}\\
\textcolor{orange}{|X_r|>1}
\end{align*}
\end{frame}
\begin{frame}{Transforming formula into ILP}
\centering $\big\downarrow$
\begin{align*}
\textcolor{red}{-5\cdot |X_s|+3\cdot |X_r|\leq 0}\\
\textcolor{blue}{5\cdot |X_s|-3\cdot |X_r|\leq 0}\\
\textcolor{orange}{|X_r|>1}
\end{align*}
\centering $\big\downarrow$
\begin{align*}
\textcolor{red}{-5\cdot |X_s|+3\cdot |X_r|+I_1 = 0}\\
\textcolor{blue}{5\cdot |X_s|-3\cdot |X_r|+I_2= 0}\\
\textcolor{orange}{|X_r|-I_3=2}
\end{align*}
\end{frame}
\begin{frame}{Transforming formula into ILP}
\centering $\big\downarrow$
\begin{align*}
\textcolor{red}{-5\cdot |X_s|+3\cdot |X_r|+I_1 = 0}\\
\textcolor{blue}{5\cdot |X_s|-3\cdot |X_r|+I_2= 0}\\
\textcolor{orange}{|X_r|-I_3=2}
\end{align*}
\centering $\big\downarrow$
\begin{align*}
\textcolor{red}{-5\cdot |v_1\cup v_3|+3\cdot |v_1\cup v_2|+I_1 = 0}\\
\textcolor{blue}{5\cdot |v_1\cup v_3|-3\cdot |v_1\cup v_2|+I_2= 0}\\
\textcolor{orange}{|v_1\cup v_2|-I_3=2}
\end{align*}
\end{frame}
\begin{frame}{ILP}
\renewcommand{\kbldelim}{(}% Left delimiter
\renewcommand{\kbrdelim}{)}% Right delimiter
\[
  \kbordermatrix{
    & v_1 & v_2 & v_3 & v_4 & I_1 & I_2 & I_3\\
    -5\cdot |X_s|+2\cdot |X_r|+I_1 & -5+3 & 3 & -5 & 0 & 1 & 0 & 0\\
5\cdot |X_s|-2\cdot |X_r|+I_2 & 5-3 & -3 & 5 & 0 & 0 & 1 & 0\\
    |X_r|-I_3 & 1 & 1 & 0 & 0 & 0 & 0 & -1\\
    } \cdot  \left( \begin{array}{c}
x_1 \\
x_2 \\
x_3 \\
x_4 \\
x_5 \\
x_6 \\
x_7 \end{array} \right)
\]\\
\centering $= \left( \begin{array}{c}
0\\0\\2
\end{array} \right)$
\end{frame}
\begin{frame}\frametitle{Upper bound}
Christos H. Papadimitriou\footnote{\tiny{On the Complexity of Integer Programming. J. ACM,28(4):765-768, Oct. 1981.}}:\\
\vspace*{1cm}
\centering
For each ILP $Ax=b$, $A$ $m\times n$ matrix, $x\in \mathbb{N}^n$ and $b\in\mathbb{R}^m$ there exists an upper bound $M$ and a solution $x^\prime\in\{0,\dots,M\}^n$ such that 
\begin{align*}
Ax^\prime =b
\end{align*}
\end{frame}
\section{Tableau for $\mathcal{ALCSCC}$}
\begin{frame}\frametitle{Tableau for $\mathcal{ALCSCC}$}
\centering
\large{Now we consider the rules for the tableau algorithm.}
\end{frame}
\begin{frame}\frametitle{Decomposing rules}
\begin{itemize}
\item $\sqcap$-rule:\\
If $x:A\sqcap B$ is in ABox then $x:A$ and $x:B$ must be in ABox
\item $\sqcup$-rule:\\
If $x:A\sqcup B$ is in ABox then $x:A$ or $x:B$ must be in ABox
\end{itemize}
\vspace*{1cm}
\pause
Always apply when possible (higher priority)
\end{frame}

\begin{frame}\frametitle{$successor$-rule}
\begin{tikzpicture}
\node[draw=black,ellipse](x) at (0,0) {$x$};
\node at (0,-2) {\begin{tabular}{c}
    rule not \\
    applied yet
\end{tabular}};
\node[draw=black, ellipse](set) at (3.5,0) {\begin{tabular}{c}
    $x:succ(c),$ \\
    $x:succ(d),$\\ 
    $\dots$
\end{tabular}};
\node at (3.5,-2) {\begin{tabular}{c}
    set of \\
    successor constraints
\end{tabular}};
\node[draw=black, ellipse](f) at (7.5,0) {\begin{tabular}{c}
    $|X_r|\leq |X_s|$\\
    $\wedge\dots$
\end{tabular}};
\node at (7.5,-2) {QFBAPA formula};
\draw[->] (x) -- (set);
\draw[->] (set) -- (f);
\end{tikzpicture}
\end{frame}

\begin{frame}\frametitle{$successor$-rule}
\centering
\begin{tikzpicture}
\node[draw=black, ellipse](f) at (0,0) {\begin{tabular}{c}
    $|X_r|\leq |X_s|$\\
    $\wedge\dots$
\end{tabular}};
\node (b) at (4,0) {\begin{tabular}{c}
    calculate \\
    upper bound
\end{tabular}};
\draw[->](f)--(b);
\pause
\node (a) at (8,0) {\begin{tabular}{c}
    use\\
    QFBAPA\\
    solver
\end{tabular}};
\draw[->](b)--(a);
\end{tikzpicture}
\end{frame}
\begin{frame}\frametitle{$successor$-rule}
\centering
\begin{tikzpicture}
\node(f) at (0,0) {use QFBAPA solver};
\node (b) at (-2,-2) {\begin{tabular}{c}
    If it returns \\
    unsatisfiable
\end{tabular} };
\draw[->](f)--(b);
\node (c) at (-2,-4) {add $x:\perp$};
\draw[->](b)--(c);
\pause
\node (a) at (2,-2) {\begin{tabular}{c}
    If it returns \\
    a solution
\end{tabular}};
\draw[->](f)--(a);
\node (s) at (2,-4) {add variables and its constraints};
\draw[->](a)--(s);
\end{tikzpicture}
\end{frame}
\begin{frame}\frametitle{Example: Concept}
\begin{gather}
pancake: succ(|contains\cap Milk|\geq |contains\cap Flavour|)\,\textcolor{black}{\sqcap}\nonumber 
\end{gather}
\begin{gather}
succ(|contains\cap Milk|\leq 2\cdot|contains\cap Flavour|)\,\textcolor{black}{\sqcap} \nonumber
\end{gather}
\begin{gather}
succ(|contains\cap Egg|\geq 1)\nonumber
\end{gather}
\end{frame}
\begin{frame}\frametitle{Example: Decomposing}
We use the $\sqcap$-rule twice and add \\
\pause
\begin{gather}
pancake: succ(|contains\cap Milk|\geq |contains\cap Flavour|),\nonumber 
\end{gather}
\begin{gather}
pancake: succ(|contains\cap Milk|\leq 2\cdot|contains\cap Flavour|),\nonumber
\end{gather}
\begin{gather}
pancake: succ(|contains\cap Egg|\geq 1)\nonumber 
\end{gather}
\end{frame}
\begin{frame}\frametitle{Example: $successor$-rule}
\centering
\begin{tikzpicture}
\node (s) at (0,0) {\begin{tabular}{c}
    $succ(|contains\cap Milk|\geq |contains\cap Flavour|)$\\
    $succ(|contains\cap Milk|\leq 2\cdot|contains\cap Flavour|)$\\
    $succ(|contains\cap Egg|\geq 1)$
\end{tabular} };
\pause
\node (q) at (0,-3)
{\begin{tabular}{c}
    $|X_{contains}\cap X_{Milk}|\geq |X_{contains}\cap X_{Flavour}|\,\wedge$\\
    $|X_{contains}\cap X_{Milk}|\leq 2\cdot|X_{contains}\cap X_{Flavour}|\,\wedge$\\
    $|X_{contains}\cap X_{Egg}|\geq 1$
\end{tabular} };
\draw[->] (s) -- (q);
\end{tikzpicture}
\end{frame}
\begin{frame}\frametitle{Example: ILP}
\begin{itemize}
\item 4 set variables $\rightarrow$ $2^4=16$ Venn regions\pause
\item BUT: every successor is a $contains$-successor\pause
\item Assume $Milk$, $Flavour$ and $Egg$ are disjoint
\end{itemize}
\pause
\begin{align*}
|X_{contains}\cap X_{Milk}|-|X_{contains}\cap X_{Flavour}|-I_1=0\\
|X_{contains}\cap X_{Milk}|-2\cdot|X_{contains}\cap X_{Flavour}|+I_2=0\\
|X_{contains}\cap X_{Egg}|-I_3=1
\end{align*}
\end{frame}
\begin{frame}\frametitle{Example: ILP}
\renewcommand{\kbldelim}{(}% Left delimiter
\renewcommand{\kbrdelim}{)}% Right delimiter
\[
  \kbordermatrix{
    & X_{M} & X_{F} & X_{E} & X_{c} & I_1 & I_2 & I_3\\
    \text{Milk}>\text{Flavour} & 1 & -1 & 0 & 1 & -1 & 0 & 0\\
	\text{not to much milk} & 1 & -2 & 0 & 1 & 0 & 1 & 0\\
    \text{Eggs} & 0 & 0 & 1 & 1 & 0 & 0 & -1\\
    } \cdot  \left( \begin{array}{c}
x_1 \\
x_2 \\
x_3 \\
x_4 \\
x_5 \\
x_6 \\
x_7\end{array} \right)
\]\\
\centering $= \left( \begin{array}{c}
0\\0\\1
\end{array} \right)$
\end{frame}
\begin{frame}\frametitle{Example: Solution of the solver}
Assume the solver returns\\
\vspace*{-0.5cm}
\begin{align*}
&X_{Milk}=\{cup_m\},\\
&X_{Flavour}=\{cup_f\}\\
&X_{egg}=\{egg1,egg2\} \text{ and }\\
&X_{contains}=X_{Milk}\cup X_{Flavour}\cup X_{Egg}
\end{align*}
\pause
\vspace*{1cm}
\centering
\begin{tikzpicture}[scale=0.6,every node/.style={scale=0.6}]
\node[draw=black, ellipse] (p) at (0,0) {$pancake$};
\node[draw=black, ellipse] (m) at (-6.5,-2.5) {$cup_m$};
\node[draw=black, ellipse] (f) at (-2,-2.5) {$cup_f$};
\node[draw=black, ellipse] (1) at (2,-2.5) {$egg1$};
\node[draw=black, ellipse] (2) at (6,-2.5) {$egg2$};
\draw[->] (p) edge node[left] {$contains$} (m);
\draw[->] (p) edge node[left] {$contains$} (f);
\draw[->] (p) edge node[right] {$contains$} (1);
\draw[->] (p) edge node[right] {$contains$} (2);
\end{tikzpicture}
\end{frame}
\section{Conclusion}
\begin{frame}
\begin{itemize}
\item Tableau for $\mathcal{ALCSCC}$
\item 2ExpSpace because of upper bound
\end{itemize}
\pause
For the future:
\begin{itemize}
\item use/find smaller upper bound
\item extend tableau for whole knowledge base
\item tableau without QFBAPA solver
\end{itemize}
\end{frame}
\end{document}
