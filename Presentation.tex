\documentclass{beamer}
\usepackage[utf8]{inputenc}
\usepackage{algorithm,algorithmic} 
\usepackage{amsmath}
\usepackage{ulem}
\usepackage{ amssymb }
\usepackage{mathtools}
\usepackage{amsthm}
\usepackage{ stmaryrd }
\usepackage{ dsfont }
\theoremstyle{definition}
\newtheorem{mydef}{Definition}
\newtheorem{mylem}{Lemma}
\newtheorem{mythe}{Theorem}
\newtheorem{mycol}{Corollary}
\usepackage{color}
\usepackage{xcolor}
\usepackage{paralist}
\usepackage{tikz}
\usetikzlibrary{positioning}
\usetikzlibrary{scopes}
\usepackage{kbordermatrix}
\usetikzlibrary{shapes,snakes}
\usepackage[absolute,overlay]{textpos}
\usetheme{Warsaw}  %% Themenwahl
\newcommand{\xdownarrow}[1]{%
  {\left\downarrow\vbox to #1{}\right.\kern-\nulldelimiterspace}
}
\tikzset{cloud/.pic={
\node[cloud, cloud puffs=8.5,cloud puff arc=90, aspect=2, draw, text width=2.5cm
    ] () at (0,0) {\tikzpictext};
}}
\makeatletter
\newenvironment<>{proofs}[1][\proofname]{%
    \par
    \def\insertproofname{#1\@addpunct{.}}%
    \usebeamertemplate{proof begin}#2}
  {\usebeamertemplate{proof end}}
\makeatother
\setbeamertemplate{footline}[frame number]
\title{A Tableau algorithm for $\mathcal{ALCSCC}$}
\author{Ryny Khy}
\date{January 12, 2021}
\begin{document}
\maketitle
\frame{\tableofcontents}
\section{Introduction}
\begin{frame}\frametitle{Motivation}
\begin{itemize}
\item Reasoning in data base
\pause
\item Satiasfiability check for reasoning
\pause
\item Tableau algorithm for satisfiabilty check
\end{itemize}
\end{frame}
\begin{frame}\frametitle{Tableau Algorithm}
Main Idea:
\begin{align*}
x:A\sqcap B\sqcap \exists r.C
\end{align*}
\only<1>{
\centering
\begin{tikzpicture}
\pic (c1) at (0,0) [pic text=$x$ must be in $A$]{cloud};
\node[draw=black,ellipse] (x) at (5,1.5) {$x$};
\node (s) at (5,-1.5){};
\end{tikzpicture}
}
\only<2>{
\centering
\begin{tikzpicture}
\pic (c1) at (0,0) [pic text=$x$ must be in $A$]{cloud};
\node[draw=black,ellipse] (x) at (5,1.5) {$x$};
\node at (6,1.5) {$A$};
\node (s) at (5,-1.5){};
\end{tikzpicture}
}
\only<3>{
\centering
\begin{tikzpicture}
\pic (c1) at (0,0) [pic text=$x$ must be in $B$]{cloud};
\node[draw=black,ellipse] (x) at (5,1.5) {$x$};
\node at (6,1.5) {$A$};
\node (s) at (5,-1.5){};
\end{tikzpicture}
}
\only<4>{
\centering
\begin{tikzpicture}
\pic (c1) at (0,0) [pic text=$x$ must be in $B$]{cloud};
\node[draw=black,ellipse] (x) at (5,1.5) {$x$};
\node at (6,1.5) {$A,B$};
\node (s) at (5,-1.5){};
\end{tikzpicture}
}
\only<5>{
\centering
\begin{tikzpicture}
\pic (c1) at (0,0) [pic text=$x$ must have an $r$-successor in $C$]{cloud};
\node[draw=black,ellipse] (x) at (5,1.5) {$x$};
\node at (6,1.5) {$A,B$};
\node(y) at (5,-1.5){};
\end{tikzpicture}
}
\only<6>{
\centering
\begin{tikzpicture}
\pic (c1) at (0,0) [pic text=$x$ must have an $r$-successor in $C$]{cloud};
\node[draw=black,ellipse] (x) at (5,1.5) {$x$};
\node at (6,1.5) {$A,B$};
\node[draw=black,ellipse] (y) at (5,-1.5){$y$};
\node at (6,-1.5) {$C$};
\draw[->] (x) edge node[left] {$r$} (y);
\end{tikzpicture}
}
\end{frame}
\begin{frame}\frametitle{Goal}
\centering
Tableau algorithm for $\mathcal{ALCSCC}$ concepts
\end{frame}
\begin{frame}\frametitle{$\mathcal{ALCSCC}$: successors}
\begin{center}
\begin{tikzpicture}
\node[draw=black,cross out] at (0,0) {$\exists r.C$};
\node[draw=black,cross out] at (2.5,0) {$\forall r.C$};
\node[draw=black,cross out] at (5.25,0) {$\leq\,m\, r.C$};
\node[draw=black,cross out] at (8.25,0) {$\geq\,m\, r.C$};
\node[draw=black] at (4,-1.75){$succ(c)$};
\end{tikzpicture}
\end{center}
$c$: \textbf{set constraint} or \textbf{cardinality constraint}
\end{frame}
\begin{frame}\frametitle{$\mathcal{ALCSCC}$: constraints}
\begin{minipage}[t]{0.5\textwidth}
\raggedright
\textbf{set constraint:}
\begin{itemize}
\item $r\subseteq s$
\item $C\cap r\subseteq D$
\item $succ(C\cap r)\subseteq succ(D)$
\end{itemize}
\end{minipage}%
\begin{minipage}[t]{0.5\textwidth}
\raggedright
\textbf{cardinality constraint}
\begin{itemize}
\item $2\,dvd\,|r|$
\item $|C\cap r|\leq |D|$
\item $|succ(C\cap r)|\leq |succ(D)|$
\end{itemize}
\end{minipage}
\end{frame}
\section{Difficulties with $\mathcal{ALCSCC}$}
\begin{frame}\frametitle{Problem with successors constraints}
\begin{align*}
x:\, succ(|s|>1)\,\sqcap\, succ(|r|\leq|s|)\, \sqcap\,succ(|r|>|s|)
\end{align*}
\pause
\begin{itemize}
\item endless loop of adding $r$- and $s$-successors
\pause
\item blocking?
\end{itemize}
\end{frame}
\begin{frame}\frametitle{Problem with blocking}
\only<1>{
\begin{align*}
x:succ(2\cdot |r|\leq 5\cdot|s|)\,\sqcap\,succ(5\cdot|s|\leq 2\cdot|r|)\,\sqcap\,succ(|r|>1)
\end{align*}
\begin{center}
\begin{tikzpicture}
\node[draw=black,ellipse] (s) at (0,0) {0};
\node[draw=black,ellipse] (r) at (4,0) {0};
\node[above = 0.3cm of s] {$s$-successors};
\node[above = 0.3cm of r] {$r$-successors};
\end{tikzpicture}
\end{center}
}
\only<2>{
\begin{align*}
x:succ(2\cdot |r|\leq 5\cdot|s|)\,\sqcap\,succ(5\cdot|s|\leq 2\cdot|r|)\,\sqcap\,\underline{succ(|r|>1)}
\end{align*}
\begin{center}
\begin{tikzpicture}
\node[draw=black,ellipse] (s) at (0,0) {0};
\node[draw=black,ellipse] (r) at (4,0) {1};
\node[above = 0.3cm of s] {$s$-successors};
\node[above = 0.3cm of r] {$r$-successors};
\end{tikzpicture}
\end{center}
}
\only<3>{
\begin{align*}
x:\underline{succ(2\cdot |r|\leq 5\cdot|s|)}\,\sqcap\,succ(5\cdot|s|\leq 2\cdot|r|)\,\sqcap\,succ(|r|>1)
\end{align*}
\begin{center}
\begin{tikzpicture}
\node[draw=black,ellipse] (s) at (0,0) {1};
\node[draw=black,ellipse] (r) at (4,0) {1};
\node[above = 0.3cm of s] {$s$-successors};
\node[above = 0.3cm of r] {$r$-successors};
\end{tikzpicture}
\end{center}
}
\end{frame}
\begin{frame}\frametitle{QFBAPA formula and solver}
\begin{align*}
x:\textcolor{red}{succ(2\cdot |r|\leq 5\cdot|s|)}\,\sqcap\,\textcolor{blue}{succ(5\cdot|s|\leq 2\cdot|r|)}\,\sqcap\,\textcolor{orange}{succ(|r|>1)}
\end{align*}
\pause
\centering $\Big\downarrow$
\begin{align*}
\textcolor{red}{2\cdot |X_r|\leq 5\cdot |X_s|}\wedge \textcolor{blue}{5\cdot |X_s|\leq 2\cdot |X_r|}\wedge \textcolor{orange}{|X_r|>1}
\end{align*}
\pause
\centering $\Big\downarrow$
\begin{align*}
X_r=\{r_1,r_2,\dots,r_5\}\text{ and }X_s=\{s_1,s_2\}
\end{align*}
\end{frame}
\section{Dealing with successor constraints}
\begin{frame}\frametitle{QFBAPA formula and solver}
What about
\begin{align*}
\textcolor{red}{2\cdot |X_r|\leq 5\cdot |X_s|}\wedge \textcolor{blue}{5\cdot |X_s|\leq 2\cdot |X_r|}\wedge \textcolor{orange}{|X_r|>1}
\end{align*}
\centering $\Big\downarrow$
\begin{align*}
|X_r|=10\text{ and }|X_s|=4
\end{align*}
\end{frame}
\begin{frame}\frametitle{QFBAPA formula and solver}
or
\begin{align*}
\textcolor{red}{2\cdot |X_r|\leq 5\cdot |X_s|}\wedge \textcolor{blue}{5\cdot |X_s|\leq 2\cdot |X_r|}\wedge \textcolor{orange}{|X_r|>1}
\end{align*}
\centering $\Big\downarrow$
\begin{align*}
|X_r|=100\text{ and }|X_s|=40
\end{align*}
\pause
$\rightarrow$ upper bound
\end{frame}
\begin{frame}\frametitle{ILP and upper bound}
\begin{align*}
\textcolor{red}{2\cdot |X_r|\leq 5\cdot |X_s|}\wedge \textcolor{blue}{5\cdot |X_s|\leq 2\cdot |X_r|}\wedge \textcolor{orange}{|X_r|>1}
\end{align*}
\centering $\big\downarrow$
\begin{align*}
\textcolor{red}{-5\cdot |X_s|+2\cdot |X_r|\leq 0}\\
\textcolor{blue}{5\cdot |X_s|-2\cdot |X_r|\leq 0}\\
\textcolor{orange}{|X_r|>1}
\end{align*}
\pause
\centering $\big\downarrow$
\begin{align*}
\textcolor{red}{-5\cdot |X_s|+2\cdot |X_r|+I_1 = 0}\\
\textcolor{blue}{5\cdot |X_s|-2\cdot |X_r|+I_2= 0}\\
\textcolor{orange}{|X_r|-I_3=1}
\end{align*}
\end{frame}
\begin{frame}\frametitle{ILP and upper bound}
\begin{itemize}
\item[]\textbf{Problem}: Are the variables disjoint or not?\pause
\item[]\textbf{Solution}: Venn region
\end{itemize}
\centering
\begin{tikzpicture}
% left hand
\scope
\clip
      (1.5,0) circle (1.5);
 (0,0) circle (1.5);
\endscope
% right hand
\scope
\clip
      (0,0) circle (1.5);
 (1.5,0) circle (1.5);
\endscope
% outline
\draw (0,0) circle (1.5) (0,1.5)  node [text=black,above = 0.1cm] {$X_s$}
      (1.5,0) circle (1.5) (1.5,1.5)  node [text=black,above = 0.1cm] {$X_r$} ;
\end{tikzpicture}
\end{frame}
\begin{frame}{Venn regions}
Two variables $X_s$ and $X_r$ $\rightarrow$ four Venn regions:
\begin{align*}
&v_1=X_s\cap X_r\\
&v_2=X_s^\neg\cap X_r\\
&v_3=X_s\cap X_r^\neg\\
&v_4=X_s^\neg\cap X_r^\neg
\end{align*}
\pause
$X_s=v_1\cup v_3$ and $X_r=v_1\cup v_2$
\end{frame}
\begin{frame}{Venn regions}
Franz Baader \footnote{\tiny{A New Description Logic with Set Constraints and Cardinality
Constraints on Role Successors. Springer International Publishing: 43-59, 2017}}:\\
\vspace*{0.5cm}
For every QFBAPA formula $\phi$ there is a number $N$, which is polynomial in the size of $\phi$ and can be computed in polynomial time such that for every solution $\sigma$ of $\phi$ there exists a solution $\sigma^\prime$ of $\phi$ such that:
\begin{itemize}
\item $|\{v|v\text{ is a Venn region and }\sigma^\prime(v)\neq \emptyset\}|\leq N$
\item $\{v|v\text{ is a Venn region and }\sigma^\prime(v)\neq \emptyset\}\subseteq \{v|v\text{ is a Venn region and }\sigma(v)\neq \emptyset\}$
\end{itemize}
\end{frame}
\begin{frame}{ILP}
\renewcommand{\kbldelim}{(}% Left delimiter
\renewcommand{\kbrdelim}{)}% Right delimiter
\[
  \kbordermatrix{
    & v_1 & v_2 & v_3 & v_4 & I_1 & I_2 & I_3\\
    -5\cdot |X_s|+2\cdot |X_r|+I_1 & -5+2 & 2 & -5 & 0 & 1 & 0 & 0\\
5\cdot |X_s|-2\cdot |X_r|+I_2 & 5-3 & -2 & 5 & 0 & 0 & 1 & 0\\
    |X_r|-I_3 & 1 & 1 & 0 & 0 & 0 & 0 & -1\\
    } \cdot  \left( \begin{array}{c}
x_1 \\
x_2 \\
x_3 \\
x_4 \\
x_5 \\
x_6 \\
x_7 \end{array} \right)
\]\\
\centering $= \left( \begin{array}{ccc}
0 & 0 & 1
\end{array} \right)$
\end{frame}
\begin{frame}\frametitle{ILP and upper bound}
Christos H. Papadimitriou\footnote{\tiny{On the Complexity of Integer Programming. J. ACM,28(4):765-768, Oct. 1981.}}:\\
\vspace*{1cm}
\centering
There exists an upper bound $M$ and $\alpha\in\{0,\dots ,M\}^n$ such that for each ILP $Ax=b$, $A$ a $m\times n$ matrix, $x\in \mathbb{N}^n$ and $b\in\mathbb{R}^m$ there exists a solution $x^\prime\in \{0,\dots ,M\}^n$ such that:
\begin{align*}
x-\alpha=x^\prime
\end{align*}
\end{frame}
\section{Tableau for $\mathcal{ALCSCC}$}
\begin{frame}\frametitle{Decomposing rules}
\begin{itemize}
\item $\sqcap$-rule:\\
If $x:A\sqcap B$ is in ABox then $x:A$ and $x:B$ must be in ABox
\item $\sqcup$-rule:\\
If $x:A\sqcup B$ is in ABox then $x:A$ or $x:B$ must be in ABox
\end{itemize}
\vspace*{1cm}
\pause
Always apply when possible (higher priority)
\end{frame}

\begin{frame}\frametitle{$successor$-rule:}
\begin{tikzpicture}
\node[draw=black,ellipse](x) at (0,0) {$x$};
\node at (0,-2) {\begin{tabular}{c}
    rule not \\
    applied yet
\end{tabular}};
\node[draw=black, ellipse](set) at (3.5,0) {\begin{tabular}{c}
    $x:succ(c),$ \\
    $x:succ(d),$\\ 
    $\dots$
\end{tabular}};
\node at (3.5,-2) {\begin{tabular}{c}
    set of \\
    successor constraints
\end{tabular}};
\node[draw=black, ellipse](f) at (7.5,0) {\begin{tabular}{c}
    $|X_r|\leq |X_s|$\\
    $\wedge\dots$
\end{tabular}};
\node at (7.5,-2) {QFBAPA formula};
\draw[->] (x) -- (set);
\draw[->] (set) -- (f);
\end{tikzpicture}
\end{frame}

\begin{frame}\frametitle{$successor$-rule}
\centering
\begin{tikzpicture}
\node[draw=black, ellipse](f) at (0,0) {\begin{tabular}{c}
    $|X_r|\leq |X_s|$\\
    $\wedge\dots$
\end{tabular}};
\node (b) at (4,0) {\begin{tabular}{c}
    calculate \\
    upper bound
\end{tabular}};
\draw[->](f)--(b);
\pause
\node (a) at (8,0) {\begin{tabular}{c}
    use\\
    QFBAPA\\
    solver
\end{tabular}};
\draw[->](b)--(a);
\end{tikzpicture}
\end{frame}
\begin{frame}\frametitle{$successor$-rule}
\centering
\begin{tikzpicture}
\node(f) at (0,0) {use QFBAPA solver};
\node (b) at (-2,-2) {If it returns unsatisfiable};
\draw[->](f)--(b);
\node (c) at (-2,-4) {add $x:\perp$};
\draw[->](b)--(c);
\pause
\node (a) at (2,-2) {If it returns a solution};
\draw[->](f)--(a);
\node (s) at (2,-4) {add variables and its constraints};
\draw[->](a)--(s);
\end{tikzpicture}
\end{frame}
\begin{frame}\frametitle{Example Concept}
\begin{gather}
fruitcake: succ(|need\cap Cake|\geq 1)\sqcap\nonumber 
\end{gather}
\begin{gather}
succ(|need\cap succ(|is\cap (Strawberry\cup 
Tangerine\cup Apple)|\geq1)|\geq1)\nonumber
\end{gather}
\end{frame}
\begin{frame}\frametitle{Decomposing}
Apply $\sqcap$-rule first:\\
\begin{gather}
fruitcake: \textcolor{blue}{succ(|need\cap Cake|\geq 1)}\,\sqcap\nonumber 
\end{gather}
\begin{gather}
\textcolor{red}{succ(|need\cap succ(|is\cap (Strawberry\cup 
Tangerine\cup Apple)|\geq1)|\geq1)},\nonumber
\end{gather}
\vspace{-0.5cm}
\pause
\begin{gather}
\textcolor{blue}{fruitcake:succ(|need\cap Cake|\geq 1)},\textcolor{red}{ fruitcake:succ(|need\cap }\nonumber
\end{gather}
\begin{gather}
\textcolor{red}{succ(|is\cap (Strawberry\cup 
Tangerine\cup Apple)|\geq 1)|\geq1)}\nonumber
\end{gather}
\end{frame}
\begin{frame}\frametitle{Consider successor constraints}
\small{Let $c=|is\cap(Strawberry\cup Tangerine\cup Apple)|\geq 1$}\\
\vspace*{0.1cm}
\begin{figure}[H]
\centering
\begin{tikzpicture}
\node[draw=black, ellipse](set) at (0,0) {\begin{tabular}{c}
    $fruitcake:succ(|need\cap Cake|\geq 1),$ \\
    $fruitcake:succ(|need\cap succ(c)|\geq1)$\\ 
\end{tabular}};
\end{tikzpicture}
\end{figure}
\vspace*{-0.1cm}
\pause
\noindent
\centering $\Big\downarrow$
\vspace*{-0.1cm}
\begin{figure}[H]
\centering
\begin{tikzpicture}
\node[draw=black, ellipse](set) at (0,0) {\begin{tabular}{c}
    $|X_{need}\cap X_{Cake}|\geq 1 \wedge$ \\
    $ |X_{need}\cap X_{succ(c)}|\geq 1$
\end{tabular}};
\end{tikzpicture}
\end{figure}
\end{frame}
\begin{frame}\frametitle{Consider successor constraints}
Assume solver returns:\\
$X_{need}\cap X_{Cake}=\{cake\}$ and $X_{need}\cap X_{succ(c)}=\{fruit\}$\\
\vspace*{0.5cm}
\pause
\centering
\begin{tikzpicture}
\node[draw=black, ellipse] (f) at (0,0) {$fruitcake$};
\node[draw=black, ellipse] (c) at (-1.5,-2) {$cake$};
\node at (-3,-2) {$Cake$};
\node[draw=black, ellipse] (i) at (1.5,-2) {$fruit$};
\node at (3,-2) {$succ(c)$};
\draw[->] (f) edge node[left] {$need$} (c);
\draw[->] (f) edge node[right] {$need$} (i);
\end{tikzpicture}
\end{frame}
\begin{frame}\frametitle{Consider next individual name $fruit$}
Same procedure for $fruit:succ(c)$:\\
\pause
\vspace*{1cm}
\begin{minipage}[t]{0.5\textwidth}
\begin{tikzpicture}[scale=0.7, every node/.style={scale=0.7}]
\node[draw=black, ellipse] (f) at (0,0) {$fruitcake$};
\node[draw=black, ellipse] (c) at (-1,-2) {$cake$};
\node at (-2.5,-2) {$Cake$};
\node[draw=black, ellipse] (i) at (1,-2) {$fruit$};
\node at (2.5,-2) {$succ(c)$};
\draw[->] (f) edge node[left] {$need$} (c);
\draw[->] (f) edge node[right] {$need$} (i);
\node[draw=black, ellipse] (s) at (1,-4) {$strawberry$};
\node at (1,-5) {$Strawberry$};
\draw[->] (i) edge node[right] {$is$} (s);
\end{tikzpicture}
\end{minipage}%
\pause
\begin{minipage}[t]{0.5\textwidth}
\begin{tikzpicture}[scale=0.7, every node/.style={scale=0.7}]
\node[draw=black, ellipse] (f) at (5,0) {$fruitcake$};
\node[draw=black, ellipse] (c) at (4,-2) {$cake$};
\node at (2.5,-2) {$Cake$};
\node[draw=black, ellipse] (i) at (6,-2) {$fruit$};
\node at (7.5,-2) {$succ(c)$};
\draw[->] (f) edge node[left] {$need$} (c);
\draw[->] (f) edge node[right] {$need$} (i);
\node[draw=black, ellipse] (s) at (4.5,-4) {$strawberry$};
\node at (4.5,-5) {$Strawberry$};
\draw[->] (i) edge node[right] {$is$} (s);
\node[draw=black, ellipse] (a) at (8,-4) {$apple$};
\node at (8,-5) {$Apple$};
\draw[->] (i) edge node[right] {$is$} (a);
\end{tikzpicture}
\end{minipage}
\end{frame}
\section{Conclusion}
\begin{frame}
\begin{itemize}
\item Tableau for $\mathcal{ALCSCC}$
\item 2ExpSpace because of upper bound
\end{itemize}
\pause
For the future:
\begin{itemize}
\item use/find smaller upper bound
\item extend tableau for whole knowledge base
\item tableau without QFBAPA solver
\end{itemize}
\end{frame}
\end{document}
