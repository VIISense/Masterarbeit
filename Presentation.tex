\documentclass{beamer}
\usepackage[utf8]{inputenc}
\usepackage{algorithm,algorithmic} 
\usepackage{amsmath}
\usepackage{ulem}
\usepackage{ amssymb }
\usepackage{mathtools}
\usepackage{amsthm}
\usepackage{ stmaryrd }
\usepackage{ dsfont }
\theoremstyle{definition}
\newtheorem{mydef}{Definition}
\newtheorem{mylem}{Lemma}
\newtheorem{mythe}{Theorem}
\newtheorem{mycol}{Corollary}
\usepackage{color}
\usepackage{xcolor}
\usepackage{paralist}
\usepackage{tikz}
\usetikzlibrary{positioning}
\usetikzlibrary{scopes}
\usepackage{kbordermatrix}
\usetikzlibrary{shapes,snakes}
\usepackage[absolute,overlay]{textpos}
\usetheme{Warsaw}  %% Themenwahl
\newcommand{\xdownarrow}[1]{%
  {\left\downarrow\vbox to #1{}\right.\kern-\nulldelimiterspace}
}
\tikzset{cloud/.pic={
\node[cloud, cloud puffs=8.5,cloud puff arc=90, aspect=2, draw, text width=2.5cm
    ] () at (0,0) {\tikzpictext};
}}
\makeatletter
\newenvironment<>{proofs}[1][\proofname]{%
    \par
    \def\insertproofname{#1\@addpunct{.}}%
    \usebeamertemplate{proof begin}#2}
  {\usebeamertemplate{proof end}}
\makeatother
\setbeamertemplate{footline}[frame number]
\title{A Tableau algorithm for $\mathcal{ALCSCC}$}
\author{Ryny Khy}
\date{\today}
\begin{document}
\maketitle
\frame{\tableofcontents}
\section{Introduction}
\begin{frame}\frametitle{Tableau Algorithm}
Main Idea:
\begin{align*}
x:A\sqcap B\sqcap \exists r.C
\end{align*}
\begin{tikzpicture}
\pic (c1) at (0,0) [pic text=$x$ must be in $A$]{cloud};
\pic (c2) at (6.5,0) [pic text=$x$ must have an $r$-successor in $C$]{cloud};
\pic (c2) at (3.1,-1.5) [pic text=$x$ must be in $B$]{cloud};
\end{tikzpicture}
\end{frame}
\begin{frame}\frametitle{Tableau Algorithm}
Main Idea:
\begin{align*}
x:A\sqcap \neg A\sqcap \exists r.C
\end{align*}
\begin{tikzpicture}
\pic (c1) at (0,0) [pic text=$x$ must be in $A$]{cloud};
\pic (c2) at (6.5,0) [pic text=$x$ must have an $r$-successor in $C$]{cloud};
\pic (c2) at (3.1,-1.5) [pic text=$x$ must not be in $B$]{cloud};
\end{tikzpicture}
\end{frame}
\begin{frame}\frametitle{$\mathcal{ALCSCC}$: successors}
\begin{center}
\begin{tikzpicture}
\node[draw=black,cross out] at (0,0) {$\exists r.C$};
\node[draw=black,cross out] at (2.5,0) {$\forall r.C$};
\node[draw=black,cross out] at (5.25,0) {$\leq\,m\, r.C$};
\node[draw=black,cross out] at (8.25,0) {$\geq\,m\, r.C$};
\node[draw=black] at (4,-1.75){$succ(c)$};
\end{tikzpicture}
\end{center}
$c$: \textbf{set constraint} or \textbf{cardinality constraint}
\end{frame}
\begin{frame}\frametitle{$\mathcal{ALCSCC}$: constraints}
\begin{minipage}[t]{0.5\textwidth}
\raggedright
\textbf{set constraint:}
\begin{itemize}
\item $r\subseteq s$
\item $C\cap r\subseteq D$
\item $succ(C\cap r)\subseteq succ(D)$
\end{itemize}
\end{minipage}%
\begin{minipage}[t]{0.5\textwidth}
\raggedright
\textbf{cardinality constraint}
\begin{itemize}
\item $2\,dvd\,|r|$
\item $|C\cap r|\leq |D|$
\item $|succ(C\cap r)|\leq |succ(D)|$
\end{itemize}
\end{minipage}
\end{frame}
\section{Tableau for $\mathcal{ALCSCC}$}
\begin{frame}\frametitle{Problem with successors constraints}
\only<1>{
\begin{align*}
x:\, succ(|s|>1)\,\sqcap\, succ(|r|=|s|)\, \sqcap\,succ(|r|>|s|)
\end{align*}
\begin{center}
\begin{tikzpicture}
\node[draw=black,ellipse] (s) at (0,0) {0};
\node[draw=black,ellipse] (r) at (4,0) {0};
\node[above = 0.3cm of s] {$s$-successors};
\node[above = 0.3cm of r] {$r$-successors};
\end{tikzpicture}
\end{center}
}
\only<2>{
\begin{align*}
x:\, \underline{succ(|s|>1)}\,\sqcap\, succ(|r|=|s|)\, \sqcap\,succ(|r|>|s|)
\end{align*}
\begin{center}
\begin{tikzpicture}
\node[draw=black,ellipse] (s) at (0,0) {1};
\node[draw=black,ellipse] (r) at (4,0) {0};
\node[above = 0.3cm of s] {$s$-successors};
\node[above = 0.3cm of r] {$r$-successors};
\end{tikzpicture}
\end{center}
}
\only<3>{
\begin{align*}
x:\, succ(|s|>1)\,\sqcap\, \underline{succ(|r|=|s|)}\, \sqcap\,succ(|r|>|s|)
\end{align*}
\begin{center}
\begin{tikzpicture}
\node[draw=black,ellipse] (s) at (0,0) {1};
\node[draw=black,ellipse] (r) at (4,0) {1};
\node[above = 0.3cm of s] {$s$-successors};
\node[above = 0.3cm of r] {$r$-successors};
\end{tikzpicture}
\end{center}
}
\only<4>{
\begin{align*}
x:\, succ(|s|>1)\,\sqcap\, succ(|r|=|s|)\, \sqcap\,\underline{succ(|r|>|s|)}
\end{align*}
\begin{center}
\begin{tikzpicture}
\node[draw=black,ellipse] (s) at (0,0) {1};
\node[draw=black,ellipse] (r) at (4,0) {2};
\node[above = 0.3cm of s] {$s$-successors};
\node[above = 0.3cm of r] {$r$-successors};
\end{tikzpicture}
\end{center}
}
\end{frame}
\begin{frame}\frametitle{Problem with blocking}
\only<1>{
\begin{align*}
x:succ(2\cdot |r|\leq 5\cdot|s|)\,\sqcap\,succ(5\cdot|s|\leq 2\cdot|r|)\,\sqcap\,succ(|r|>1)
\end{align*}
\begin{center}
\begin{tikzpicture}
\node[draw=black,ellipse] (s) at (0,0) {0};
\node[draw=black,ellipse] (r) at (4,0) {0};
\node[above = 0.3cm of s] {$s$-successors};
\node[above = 0.3cm of r] {$r$-successors};
\end{tikzpicture}
\end{center}
}
\only<2>{
\begin{align*}
x:succ(2\cdot |r|\leq 5\cdot|s|)\,\sqcap\,succ(5\cdot|s|\leq 2\cdot|r|)\,\sqcap\,\underline{succ(|r|>1)}
\end{align*}
\begin{center}
\begin{tikzpicture}
\node[draw=black,ellipse] (s) at (0,0) {0};
\node[draw=black,ellipse] (r) at (4,0) {1};
\node[above = 0.3cm of s] {$s$-successors};
\node[above = 0.3cm of r] {$r$-successors};
\end{tikzpicture}
\end{center}
}
\only<3>{
\begin{align*}
x:\underline{succ(2\cdot |r|\leq 5\cdot|s|)}\,\sqcap\,succ(5\cdot|s|\leq 2\cdot|r|)\,\sqcap\,succ(|r|>1)
\end{align*}
\begin{center}
\begin{tikzpicture}
\node[draw=black,ellipse] (s) at (0,0) {1};
\node[draw=black,ellipse] (r) at (4,0) {1};
\node[above = 0.3cm of s] {$s$-successors};
\node[above = 0.3cm of r] {$r$-successors};
\end{tikzpicture}
\end{center}
}
\end{frame}
\begin{frame}\frametitle{QFBAPA formula and solver}
\begin{align*}
x:\textcolor{red}{succ(2\cdot |r|\leq 5\cdot|s|)}\,\sqcap\,\textcolor{blue}{succ(5\cdot|s|\leq 2\cdot|r|)}\,\sqcap\,\textcolor{orange}{succ(|r|>1)}
\end{align*}
\pause
\centering $\Big\downarrow$
\begin{align*}
\textcolor{red}{2\cdot |X_r|\leq 5\cdot |X_s|}\wedge \textcolor{blue}{5\cdot |X_s|\leq 2\cdot |X_r|}\wedge \textcolor{orange}{|X_r|>1}
\end{align*}
\pause
\centering $\Big\downarrow$
\begin{align*}
|X_r|=5\text{ and }|X_s|=2
\end{align*}
\end{frame}
\begin{frame}\frametitle{QFBAPA formula and solver}
What about
\begin{align*}
\textcolor{red}{2\cdot |X_r|\leq 5\cdot |X_s|}\wedge \textcolor{blue}{5\cdot |X_s|\leq 2\cdot |X_r|}\wedge \textcolor{orange}{|X_r|>1}
\end{align*}
\centering $\Big\downarrow$
\begin{align*}
|X_r|=10\text{ and }|X_s|=4
\end{align*}
\end{frame}
\begin{frame}\frametitle{QFBAPA formula and solver}
What about
\begin{align*}
\textcolor{red}{2\cdot |X_r|\leq 5\cdot |X_s|}\wedge \textcolor{blue}{5\cdot |X_s|\leq 2\cdot |X_r|}\wedge \textcolor{orange}{|X_r|>1}
\end{align*}
\centering $\Big\downarrow$
\begin{align*}
|X_r|=100\text{ and }|X_s|=40
\end{align*}
\end{frame}
\begin{frame}\frametitle{QFBAPA formula and solver}
\begin{itemize}
\item[] We have infinite possible solutions\pause
\item[] Do we need all of them?\pause
\item[] No: Consider formula as ILP and calculate an upper bound
\end{itemize}
\end{frame}
\begin{frame}\frametitle{ILP and upper bound}
\begin{align*}
\textcolor{red}{2\cdot |X_r|\leq 5\cdot |X_s|}\wedge \textcolor{blue}{5\cdot |X_s|\leq 2\cdot |X_r|}\wedge \textcolor{orange}{|X_r|>1}
\end{align*}
\centering $\Big\downarrow$
\begin{align*}
\textcolor{red}{-5\cdot |X_s|+2\cdot |X_r|\leq 0}\\
\textcolor{blue}{5\cdot |X_s|-2\cdot |X_r|\leq 0}\\
\textcolor{orange}{-|X_r|<-1}
\end{align*}
\centering $\Big\downarrow$
\begin{align*}
\textcolor{red}{-5\cdot |X_s|+2\cdot |X_r|+I_1 = 0}\\
\textcolor{blue}{5\cdot |X_s|-2\cdot |X_r|+I_2= 0}\\
\textcolor{orange}{-|X_r|+I_3=-1}
\end{align*}
\end{frame}
\begin{frame}\frametitle{ILP and upper bound}
\begin{itemize}
\item[]\textbf{Problem}: Are the variables disjoint or not?\pause
\item[]\textbf{Solution}: Venn region
\end{itemize}
\centering
\begin{tikzpicture}
% left hand
\scope
\clip
      (1.5,0) circle (1.5);
 (0,0) circle (1.5);
\endscope
% right hand
\scope
\clip
      (0,0) circle (1.5);
 (1.5,0) circle (1.5);
\endscope
% outline
\draw (0,0) circle (1.5) (0,1.5)  node [text=black,above = 0.1cm] {$X_s$}
      (1.5,0) circle (1.5) (1.5,1.5)  node [text=black,above = 0.1cm] {$X_r$} ;
\end{tikzpicture}
\end{frame}
\begin{frame}{Venn regions}
Two variables $X_s$ and $X_r$ $\rightarrow$ four Venn regions:
\begin{align*}
&v_1=X_s\cap X_r\\
&v_2=X_s^\neg\cap X_r\\
&v_3=X_s\cap X_r^\neg\\
&v_4=X_s^\neg\cap X_r^\neg
\end{align*}
\pause
$X_s=v_1\cup v_3$ and $X_r=v_1\cup v_2$
\end{frame}
\begin{frame}{ILP}
\renewcommand{\kbldelim}{(}% Left delimiter
\renewcommand{\kbrdelim}{)}% Right delimiter
\[
  \kbordermatrix{
    & v_1 & v_2 & v_3 & v_4 & I_1 & I_2 & I_3\\
    -5\cdot |X_s|+2\cdot |X_r|+I_1 & -5+2 & 2 & -5 & 0 & 1 & 0 & 0\\
5\cdot |X_s|-2\cdot |X_r|+I_2 & 5-3 & -2 & 5 & 0 & 0 & 1 & 0\\
    -|X_r|+I_3 & -1 & -1 & 0 & 0 & 0 & 0 & 1\\
    } \cdot  \left( \begin{array}{c}
x_1 \\
x_2 \\
x_3 \\
x_4 \\
x_5 \\
x_6 \\
x_7 \end{array} \right)
\]\\
\centering $= \left( \begin{array}{ccc}
0 & 0 & -1
\end{array} \right)$
\end{frame}
\begin{frame}\frametitle{ILP and upper bound}
Christos H. Papadimitriou\footnote{\tiny{On the Complexity of Integer Programming. J. ACM,28(4):765-768, Oct. 1981.}}:\\
\vspace*{1cm}
\centering
There exists an upper bound $M$ such that for each ILP $Ax=b$, $A$ a $m\times n$ matrix, $x\in \mathbb{N}^n$ and $b\in\mathbb{R}^m$ there exists a solution $x^\prime\in \{0,\dots ,M\}^n$.
\end{frame}
\begin{frame}\frametitle{Tableau algorithm}
\begin{itemize}
\item $\sqcap$-rule:\\
If $x:A\sqcap B$ is in ABox then $x:A$ and $x:B$ must be in ABox
\item $\sqcup$-rule:\\
If $x:A\sqcup B$ is in ABox then $x:A$ or $x:B$ must be in ABox
\end{itemize}
\end{frame}
\begin{frame}\frametitle{Tableau algorithm}
\begin{itemize}
\item $successor$-rule:
\end{itemize}
\begin{tikzpicture}
\node[draw=black,ellipse](x) at (0,0) {$x$};
\node at (0,-2) {rule not applied yet};
\node[draw=black, ellipse, align=left](set) at (4,0) {$x:succ(c),$\\$x:succ(d),$\\\dots};
\node at (4,-2) {set of successor assertions};
\node[draw=black, ellipse, align=left](f) at (8,0) {$|X_r|\leq |X_s|$\\$\wedge\dots$};
\node at (8,-2) {QFBAPA formula};
\draw[->] (x) -- (set);
\draw[->] (set) -- (f);
\end{tikzpicture}
\end{frame}
\begin{frame}\frametitle{Tableau algorithm}
\begin{itemize}
\item $successor$-rule:
\end{itemize}
\centering
\begin{tikzpicture}
\node[draw=black, ellipse, align=left](f) at (0,0) {$|X_r|\leq |X_s|$\\$\wedge\dots$};
\node (b) at (-3,-2) {calculate upper bound};
\draw[->](f)--(b);
\pause
\node (s) at (0,-3.5) {use QFBAPA solver};
\draw[->](f)--(s);
\pause
\node (a) at (3,-2) {add variables accordingly};
\draw[->](f)--(a);
\end{tikzpicture}
\end{frame}
\end{document}
