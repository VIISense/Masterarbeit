
\documentclass[a4paper,11pt]{scrartcl}
\usepackage[english]{babel}  % falls der Artikel auf Deutsch verfasst ist
% Verwenden Sie nur EINE der beiden folgenden Zeilen, je nachdem, ob Ihr
% Betriebssystem LATIN1 (=ISO-8859-1) oder UTF8 als Zeichenkodierung
% verwendet. Ob Sie die richtige verwenden, merken Sie daran, dass
% die Umlaute richtig im Dokument dargestellt werden.
\usepackage[utf8]{inputenc}
\usepackage{algorithm,algorithmic} 
\usepackage{amsmath}
\usepackage{ulem}
\usepackage{amssymb}
\usepackage{mathtools}
\usepackage{amsthm}
\usepackage{stmaryrd}
\usepackage{dsfont}
\theoremstyle{definition}
\newtheorem{mydef}{Definition}
\newtheorem{mylem}{Lemma}
\newtheorem{mythe}{Theorem}
\newtheorem{mycol}{Corollary}
\newtheorem{mypro}{Proposition}
\usepackage{color}
\usepackage{paralist}
\usepackage{tikz}
\usetikzlibrary{shapes,snakes,arrows}
\usepackage{verbatim}
\usetikzlibrary{positioning}
\usepackage{ulem}
\newcommand{\RM}[1]{\MakeUppercase{\romannumeral #1{}}}
\usepackage{url}
\begin{document}
\section{Preliminaries}
\begin{mydef}[$QFBAPA$]
Let $T$ be a set of symbols
\begin{itemize}
\item set terms over $T$ are:
\begin{itemize}
\item empty set $\emptyset$ and universal set $
\mathcal{U}$
\item every set symbol in $T$
\item if $s,t$ are set terms then also $s\cap t$, $s\cup t$ and $s^{\neg}$
\end{itemize}
\item set constraints over $T$ are
\begin{itemize}
\item $s\subseteq t$ and $s\not\subseteq t$
\item $s=t$ and $s\neq t$
\end{itemize}
where $s,t$ are set terms
\item cardinality terms over $T$ are:
\begin{itemize}
\item every number $n\in \mathbb{N}$
\item $|s|$ if $s$ is a set term
\item if $k,l$ are cardinality terms then also $k+l$ and $n\cdot k$, $n\in \mathbb{N}$
\end{itemize}
\item cardinality constraints over $T$ are:
\begin{itemize}
\item $k=l$ and $k\neq l$
\item $k<l$ and $k\nless l$
\item $k\leq l$ and $k\not\leq l$
\item $n$ $dvd$ $k$ and $n$ $\neg dvd$ $k$
\end{itemize}
where $k,l$ are cardinality terms and $n\in\mathbb{N}$
\end{itemize}
\end{mydef}
\begin{mydef}[$\mathcal{ALCSCC}$]
Concepts are:
\begin{itemize}
\item all concept names
\item if $C,D$ are concepts then:
\begin{itemize}
\item $\neg C$
\item $C\sqcup D$
\item $C\sqcap D$
\end{itemize}
\item $succ(c)$ if $c$ is a set or cardinality constraint over $\mathcal{ALCSCC}$ concepts and role names
\end{itemize}
\end{mydef}
The set $S$ is a set of constraints of the form $x:C$ and $(x,y):s$, where $C$ is a concept, $s$ a set term and $x,y$ variables. The constraint $(x,y):s$ denotes that $y$ is a successor of $x$, while $y$ must hold $s$.\\
For the next definition we denote with $s\in k$ that the set term $s$ occurs in the cardinality constraint $k$.
\begin{mydef}[Number of successors]
Let $S$ be a set of constraints, $x$ be a variable and $s$ be a set term. Then $\#s(x)$ is the number of successors of $x$ in $S$ which satisfy $s$:
\begin{align*}
\#s(x)=|\{s|(x,y):s\in S\}|
\end{align*}
A constraint regarding a variable $x$ is \textit{violated} if
\begin{itemize}
\item $x:succ(k\leq n)$ and $n<\sum_{s\in k}\#s(x)$
\item $x:succ(k\geq n)$ and $n>\sum_{s\in k}\#s(x)$
\item $x:succ(k\leq l)$ and $\sum_{t\in l}\#t(x)<\sum_{s\in k}\#s(x)$
\item $x:succ(k\geq l)$ and $\sum_{t\in l}\#t(x)>\sum_{s\in k}\#s(x)$
\end{itemize} 
where $n\in\mathbb{N}$ a integer.
\end{mydef}
\section{Tableau}
\begin{mydef}[Merge]
\textit{Merging} $y_1$ and $y_2$ results in one variable $y$: replace all occurrence of $y_1$ and $y_2$ with $y$. 
\end{mydef}
Note that by merging two successors other constraints might become violated:
\begin{align}
S=\{&x:succ(|r\cap A|=1)\sqcap succ(|r\cap B|=1)\sqcap succ(|r|>1),\nonumber\\ &y_1:A, y_2:B, x.r.y_1, x.r.y_2\}
\end{align}
If we merge $y_1$ and $y_2$ then the constraint $x:succ(|r|>1)$ which was satisfied becomes violated. But not only constraints regarding $x$ might become violated after merging two successors of $x$: 
\begin{align}
S=\{&x:succ(|r|\leq 1), x.r.y_1, x.r.y_2,\nonumber
\\&y_1:succ(|s|\leq 1), y_2:succ(|s|\leq 1), y_1.s.z_1, y_2.s.z_2\}
\end{align}
We see that the first constraint is violated and therefore merging $y_1$ and $y_2$ would solve the problem but on the other hand the constraints regarding $y_1$ and $y_2$ become violated: 
\begin{align*}
S=\{x:succ(|r|\leq 1), x.r.y,
y:succ(|s|\leq 1), y.s.z_1, y.s.z_2\}
\end{align*}
To solve this problem we merge $z_1$ and $z_2$.
\begin{mydef}[Clash]
A constraint set $S$ contains a \textit{clash} if
\begin{itemize}
\item $\{x:\perp\}\subseteq S$ or
\item $\{x:A,\,x:\neg A\}\subseteq S$ or
\item $\{x:succ(k\leq l)\}$ and $\#k(x)>\#l(x)$
\item $\{x:succ(k\leq n)\}$ and $\#k(x)>n$
\item $\{x:succ(k\geq l)\}$ and $\#k(x)<\#l(x)$
\item $\{x:succ(k\geq n)\}$ and $\#k(x)<n$
\item $\{x:succ(n\, dvd\, k)\}$ and $mod(\#k(x),n)\neq 0$
\end{itemize}
\end{mydef}
For the next definition we define first properties of the following notations:
\begin{itemize}
\item Conjunction binds stronger than disjunction: $s\cup t\cap u = s\cup (t\cap u)$
\item $k\leq l$ and $k\geq l$ iff $k=l$
\item $s\subseteq t$ and $s\supset t$ iff $s=t$
\end{itemize}
\begin{mydef}[Tableau]
Let $S$ be a set of constraints. 
\begin{enumerate}
\item\label{cap} $\sqcap$-rule: In $S$ is $x:C_1\sqcap C_2$ but not both $x:C_1$ and $x:C_2$\\
$\rightarrow$ $S:=S\cup\{x:C_1, x:C_2\}$
\item\label{cup} $\sqcup$-rule: In $S$ is $x:C_1\sqcup C_2$ but neither $x:C_1$ or $x:C_2$\\
$\rightarrow$ $S:=S\cup\{x:C_1\}$ or $S:=S\cup\{x:C_2\}$
\item\label{choose}$choose$-rule: In $S$ are
$x:succ(k\leq l$), $y:C$ or $x.r.y$ and\\ $C$ or $r$ occur in $k$ but $(x,y):s\not\in S$ for every set term $s$ in $k$\\
$\rightarrow$ choose one set term $s$ in $k$ and either $S:=S\cup\{(x,y):s\}$ or $S:=S\cup\{(x,y):\neg s\}$
\item\label{c}$cardinality$-rule: In $S$ is $x:succ(c)$, with $c\in\{k=l,\,k\leq l,\,k<l,\, n\, dvd l\}$, which is violated 
\begin{enumerate}
\item \label{setterm} if there is a set term $s$ in $l$\\
$\rightarrow$ introduce new variable $y$ and $S:=S\cup\{(x,y):x\}$
\item \label{exceeded} if $l\in \mathbb{N}$ does not contain a set term then merge two successor $y_1\neq y_2$ of $x$ for which $(x,y_1):k\in S$ and $(x,y_2):k\in S$ if no other constraints regarding $x$ become violated
\end{enumerate}
\item\label{s}$set$-rule: In $S$ are $x:succ(c_1\subseteq c_2)$ and $(x,y):c_1$ but not $(x,y):c_2$\\
$\rightarrow$ $S:=S\cup\{(x,y):c_2\}$
\item\label{repeat} $set.term$-rule (Repeat until inapplicable): In $S$ is $(x,y):s$ and
\begin{enumerate}
\item\label{setterm1} $s=s_1\cap s_2$ but $\{(x,y):s_1,\,(x,y):s_2\}\not\subseteq S$ then\\
$\rightarrow$ $S:=S\cup \{(x,y):s_1,\,(x,y):s_2\}$ 
\item\label{setterm2} $s=s_1\cup s_2$ and neither $\{(x,y):s_1\}\subseteq S$ nor $S\{(x,y):s_2\}\subset S$ then\\
$\rightarrow$ either $S:=S\cup \{(x,y):s_1\}$ or $S:=S\cup \{(x,y):s_2\}$ 
\item\label{setterm3} $s=r$ and $x.r.y\notin S$ then \\
$\rightarrow$ $S:=S\cup\{x.r.y\}$
\item\label{setterm4} $s=C$ and $y:C\notin S$, where $C$ is a $\mathcal{ALCSCC}$ concepts then \\
$\rightarrow$ $S:=S\cup\{y:C\}$
\end{enumerate}
\end{enumerate}
\end{mydef}
Note that:
\begin{itemize}
\item $\ref{exceeded}$ is never applicable for $n\, dvd\, l$
\item $n_1\,dvd\,n_2\cdot l$ and $n_1\,\neg dvd\,n_2$ then $n_1\,dvd\,l$ eventually
\end{itemize}
Example for $\ref{exceeded}$:
\begin{align}
S=\{x:succ(|r|=1)\sqcap succ(|r\cap s|=1)\sqcap succ=(|r\cap C|=1|\}
\end{align}
After rule \ref{cap} (two times):
\begin{align*}
S=\{x:succ(|r|=1)\sqcap succ(|r\cap s|=1)\sqcap succ=(|r\cap C|=1|\\
x:succ(|r|=1), x:succ(|r\cap s|=1), x:succ=(|r\cap C|=1|
\}
\end{align*}
If we try to satisfy at least two of the new constraints by the Tableau-algorithm above we end up with at least one constraint being violated. Let say we use the rules on the three new constraints sequentially. Then we have 
\begin{figure}[H]
\centering
\begin{tikzpicture}
\node (0) at (0,0) {$x$};
\node (1) at (-4,-2) {$y_1$};
\node (2) at (0,-2) {$y_2$};
\node (3) at (4,-2) {$y_3$};
\node[below = 1mm of 3] {$C$};
\path[->] (0) edge node[above]{$r$} (1);
\path[->] (0) edge node[left]{$r\cap s$} (2);
\path[->] (0) edge node[above]{$r$} (3);
\end{tikzpicture}
\end{figure}
After using rule \ref{exceeded} two times we have the variable $x$ and its only $r\cap s$-successor $y$ which is of the concept $C$. We could use this rule because we do not violate any other constraints. This condition helps to prevent an infinite chain of rule application:
\begin{align}
S=\{x:succ(|r|<2)\sqcap succ(|r|\geq 2)\}	
\end{align}
First we apply the rules \ref{setterm} and \ref{setterm3} two times to add two $r$-successors for $x$ hence $x:succ(|r|<2)$ is not satisfied any more. If we ignore the condition in rule \ref{exceeded} and apply it then we merge the two successors leading to $x:succ(|r|<2)$ being satisfied but $x:succ(|r|\geq 2)$ being violated. Then we apply the rules \ref{setterm} and \ref{setterm3} again leading to $x:succ(|r|<2)$ being violated again and so on. By the condition in \ref{exceeded} we can not use rule \ref{exceeded} from the beginning which means the algorithm terminates with a clash stating that the constraint set is unsatisfiable.
\end{document}