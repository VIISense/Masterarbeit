\documentclass[a4paper,11pt]{scrartcl}
\usepackage[english]{babel}  % falls der Artikel auf Deutsch verfasst ist
% Verwenden Sie nur EINE der beiden folgenden Zeilen, je nachdem, ob Ihr
% Betriebssystem LATIN1 (=ISO-8859-1) oder UTF8 als Zeichenkodierung
% verwendet. Ob Sie die richtige verwenden, merken Sie daran, dass
% die Umlaute richtig im Dokument dargestellt werden.
\usepackage[utf8]{inputenc}
\usepackage{algorithm,algorithmic} 
\usepackage{amsmath}
\usepackage{ulem}
\usepackage{amssymb}
\usepackage{mathtools}
\usepackage{amsthm}
\usepackage{stmaryrd}
\usepackage{dsfont}
\newtheoremstyle{break}
  {\topsep}{\topsep}%
  {\itshape}{}%
  {\bfseries}{}%
  {\newline}{}%
\theoremstyle{break}
\theoremstyle{definition}
\newtheorem{mydef}{Definition}
\newtheorem{mylem}{Lemma}
\newtheorem{mythe}{Theorem}
\newtheorem{mycol}{Corollary}
\newtheorem{mypro}{Proposition}
\newtheorem{ex}{Example}
\usepackage{color}
\usepackage{paralist}
\usepackage{tikz}
\usetikzlibrary{shapes,snakes,arrows}
\usepackage{verbatim}
\usetikzlibrary{positioning}
\usepackage{ulem}
\newcommand{\RM}[1]{\MakeUppercase{\romannumeral #1{}}}
\usepackage{url}
\makeatletter
\newcommand{\oset}[3][0ex]{%
  \mathrel{\mathop{#3}\limits^{
    \vbox to#1{\kern-2\ex@
    \hbox{$\scriptstyle#2$}\vss}}}}
\makeatother
\begin{document}
\section{Preliminaries}
Let $\mathbf{C}$ be a set of concept names and $\mathbf{R}$ a set of role names such that they are disjoint.
\begin{mydef}[$QFBAPA$]
Let $T$ be a set of symbols
\begin{itemize}
\item set terms over $T$ are:
\begin{itemize}
\item empty set $\emptyset$ and universal set $
\mathcal{U}$
\item every set symbol in $T$
\item if $s,t$ are set terms then also $s\cap t$, $s\cup t$ and $s^{\neg}$
\end{itemize}
\item set constraints over $T$ are
\begin{itemize}
\item $s\subseteq t$ and $s\not\subseteq t$
\item $s=t$ and $s\neq t$
\end{itemize}
where $s,t$ are set terms
\item cardinality terms over $T$ are:
\begin{itemize}
\item every number $n\in \mathbb{N}$
\item $|s|$ if $s$ is a set term
\item if $k,l$ are cardinality terms then also $k+l$ and $n\cdot k$, $n\in \mathbb{N}$
\end{itemize}
\item cardinality constraints over $T$ are:
\begin{itemize}
\item $k=l$ and $k\neq l$
\item $k<l$ and $k\geq l$
\item $k\leq l$ and $k>l$
\item $n$ $dvd$ $k$ and $n$ $\neg dvd$ $k$
\end{itemize}
where $k,l$ are cardinality terms and $n\in\mathbb{N}$
\end{itemize}
For readability we use $\lesseqgtr$ to address the comparison symbols $=,\,\leq,\,\geq,\,<,\,>$. The negation $\not\lesseqgtr$ address the symbols $\neq,\,>,\,<,\,\geq,\,\leq$ respectively.
\end{mydef}
\begin{mydef}[$\mathcal{ALCSCC}$]
Concepts are:
\begin{itemize}
\item all concept names
\item $succ(c)$ if $c$ is a set or cardinality constraint over $\mathcal{ALCSCC}$ concepts and role names
\item if $C,D$ are concepts then:
\begin{itemize}
\item $\neg C$
\item $C\sqcup D$
\item $C\sqcap D$
\end{itemize}
\end{itemize}
\end{mydef}
\begin{mydef}[Negation Normal Form]
A concept is in \textit{negation normal form} ($NNF$) if the negation sign $\neg$ appears only in front of a concept name. Let $C$ be a arbitrary concept. Its $NNF$ is obtained by applying the following rules
\begin{figure}[H]
\begin{minipage}[t]{.5\textwidth}
\raggedright
\begin{itemize}
\item $\neg\top$ $\rightarrow$ $\perp$
\item $\neg\perp$ $\rightarrow$ $\top$
\item $\neg\neg C$ $\rightarrow$ $C$
\item $\neg(C\sqcap D)$ $\rightarrow$ $\neg C \sqcup \neg D$
\item $\neg(C\sqcup D)$ $\rightarrow$ $\neg C \sqcap \neg D$
\item $\neg succ(c)$ $\rightarrow$ $succ(\neg c)$
\item $\neg (s^\neg)$ $\rightarrow$ $s$
\end{itemize}
\end{minipage}% <---------------- Note the use of "%"
\begin{minipage}[t]{.5\textwidth}
\raggedleft
\begin{itemize}
\item $\neg (c_1\subseteq c_2)$ $\rightarrow$ $c_1\subseteq \neg c_2$
\item $\neg (k\lesseqgtr l)$ $\rightarrow$ $k\not\lesseqgtr l$
\item $\neg (n\text{ } dvd \text{ } k)$ $\rightarrow$ $n\text{ } \neg dvd \text{ } k$
\item $\neg (n\text{ } \neg dvd \text{ } k)$ $\rightarrow$ $n\text{ } dvd \text{ } k$
\item $\neg (s\cap t)$ $\rightarrow$ $\neg s \cup \neg t$
\item $\neg (s\cup t)$ $\rightarrow$ $\neg s \cap \neg t$
\end{itemize}
\end{minipage}
\end{figure}
\end{mydef}
With $NNF(C)$ we denote the concept which is obtained by applying the rules above on $C$ until none is applicable any more.\\
The set $S$ is a finite set of assertions of the form $x:C$ and $(x,y):s$, where $C$ is a concept, $s$ a set term and $x,y$ variables. The set $Var(S)$ is the set of variables occurring in $S$.
\begin{mydef}[Interpretation]
An \textit{interpretation} $\mathcal{I=(\Delta^\mathcal{I},\cdot^\mathcal{I})}$ over $\mathcal{ALCSCC}$ consists of a non-empty set $\Delta^\mathcal{I}$ and a mapping $\cdot^\mathcal{I}$ which maps:
\begin{itemize}
\item $\emptyset$ to $\emptyset^\mathcal{I}$
\item $\mathcal{U}$ to $\mathcal{U}^\mathcal{I}\subseteq \Delta^\mathcal{I}$
\item every concept names $A\in\mathbf{C}$ to $A^\mathcal{I}\subseteq \Delta^\mathcal{I}$
\item every role name $r\in\mathbf{R}$ to $r^\mathcal{I}\subseteq\Delta^\mathcal{I}\times\Delta^\mathcal{I}$, such that every element in $\Delta^\mathcal{I}$ has a finite number of successors.
\end{itemize}
The set $r^\mathcal{I}(x)$ contains all elements $y$ such that $(x,y)\in r^\mathcal{I}$ e.g. it contains all $r$-successors of $x$.\\
For compound concepts the mapping $\cdot^\mathcal{I}$ is extended inductively as follows
\begin{itemize}
\item $(C\sqcap D)^\mathcal{I}:=C^\mathcal{I}\sqcap D^\mathcal{I}$, $(C\sqcup D)^\mathcal{I}:=C^\mathcal{I}\sqcup D^\mathcal{I}$
\item $(\neg C)^\mathcal{I}:=\Delta^\mathcal{I}\backslash C^\mathcal{I}$
\item $(s\cap t)^\mathcal{I}:= s^\mathcal{I}\cap t^\mathcal{I}$, $(s\cup t)^\mathcal{I}:= s^\mathcal{I}\cup t^\mathcal{I}$
\item $(s^\neg)^\mathcal{I}:=\mathcal{U}^\mathcal{I}\backslash s^\mathcal{I}$
\item $|s|^\mathcal{I}:=|s^\mathcal{I}|$
\item $(k+l)^\mathcal{I}:=(k^\mathcal{I}+l^\mathcal{I})$, $(n\cdot k)^\mathcal{I}:= n\cdot k^\mathcal{I}$
\item $succ(c)^\mathcal{I}=\{x\in \Delta^\mathcal{I}|$the mapping $\cdot^{\mathcal{I}_x}$ satisfies $c\}$
\end{itemize}
The mapping $\cdot^{\mathcal{I}_x}$ maps $\emptyset$ to $\emptyset^\mathcal{I}$, $\mathcal{U}$ to $\mathcal{U}^{\mathcal{I}_x}:=\{\bigcup_{r\in\mathbf{R}}r^\mathcal{I}(x)\}$, every concept $C$ occurring in $c$ to $C^{\mathcal{I}_x}:=C^\mathcal{I}\cap \mathcal{U}^{\mathcal{I}_x}$ and every role name $r$ occurring in $c$ to $r^{\mathcal{I}_x}:=r^\mathcal{I}$.\\
The mappings satisfies for the set terms $s,t$ and the cardinality terms $k,l$
\begin{itemize}
\item $s=t$ iff $s^\mathcal{I}=t^\mathcal{I}$
\item $s\subseteq t$ iff $s^\mathcal{I}\subseteq t^\mathcal{I}$
\item $k\lesseqgtr l$ iff $k^\mathcal{I}\lesseqgtr l^\mathcal{I}$
\item $n\,dvd\,l$ iff $\exists m\in\mathbb{N}:n\cdot m = l^\mathcal{I}$
\end{itemize}
An \textit{assignment} $\pi_\mathcal{I}: Var(S)\shortrightarrow \Delta^\mathcal{I}$ satisfies 
\begin{itemize}
\item $x:C$ iff $\pi_\mathcal{I}(x)\in C^\mathcal{I}$ 
\item $(x,y):s$ iff $(\pi_\mathcal{I}(x),\pi_\mathcal{I}(y))\in s^\mathcal{I}$
\end{itemize} 
A constraint $c$ is satisfiable if there exists an interpretation $\mathcal{I}$ and an assignment $\pi_\mathcal{I}$ such that $\pi_\mathcal{I}$ satisfies $c$. $\pi_\mathcal{I}$ satisfies a constraint set $S$ if $\pi_\mathcal{I}$ satisfies every constraint in $S$. $S$ is satisfiable if there exists an interpretation $\mathcal{I}$ and an assignment $\pi_\mathcal{I}$ such that $\pi_\mathcal{I}$ satisfies $S$.
\end{mydef}
\begin{mydef}[Number of successors]
Let $S$ be a set of constraints, $\mathcal{I}$ be an interpretation, $\pi_\mathcal{I}:Var(S)\shortrightarrow \Delta^\mathcal{I}$ be an assignment, $x$ be a variable and $k$ be a cardinality term. The number of successors of $x$ in $k$ in the interpretation $\mathcal{I}$ is denoted by $n_\mathcal{I}(x_{S_k}):=|k^\mathcal{I}\cap r^\mathcal{I}(x)|$.\\
A constraint regarding a variable $x$ in an interpretation $\mathcal{I}$ is \textit{violated} if
\begin{itemize}
\item $x:succ(k\lesseqgtr n)$ and $n_\mathcal{I}(x_{S_k})\not\lesseqgtr n$
\item $x:succ(k\lesseqgtr l)$ and $n_\mathcal{I}(x_{S_k})\not\lesseqgtr n_\mathcal{I}(x_{S_l})$
\item $x:succ(n\,dvd\,k)$ and $mod(n_\mathcal{I}(x_{S_k}),n)\neq 0$
\end{itemize} 
where $n\in\mathbb{N}$.
\end{mydef}
\section{Tableau}
For the algorithm we assume that the constraints in the constraint set are in $NNF$.
\begin{mydef}[Merge]
\textit{Merging} $y_1$ and $y_2$ results in one variable $y$: replace all occurrence of $y_1$ and $y_2$ with $y$. 
\end{mydef}
Note that by merging two successors other constraints might become violated:
\begin{align}
S=\{&x:succ(|r\cap A|=1)\sqcap succ(|r\cap B|=1)\sqcap succ(|r|>1),\nonumber\\ &y_1:A, y_2:B, x.r.y_1, x.r.y_2\}
\end{align}
If we merge $y_1$ and $y_2$ then the constraint $x:succ(|r|>1)$ which was satisfied becomes violated.\\
But not only constraints regarding $x$ might become violated after merging two successors of $x$: 
\begin{align}
S=\{&x:succ(|r|\leq 1), x.r.y_1, x.r.y_2,\nonumber
\\&y_1:succ(|s|\leq 1), y_2:succ(|s|\leq 1), y_1.s.z_1, y_2.s.z_2\}
\end{align}
We see that the first constraint is violated and therefore merging $y_1$ and $y_2$ to $y$ would solve the problem but on the other hand the constraints regarding $y$ become violated: 
\begin{align*}
S=\{x:succ(|r|\leq 1), x.r.y,
y:succ(|s|\leq 1), y.s.z_1, y.s.z_2\}
\end{align*}
To solve this problem we have to merge $z_1$ and $z_2$.\\
For the next definition we define first properties of the following notations:
\begin{itemize}
\item Conjunction binds stronger than disjunction: $s\cup t\cap u = s\cup (t\cap u)$
\item if $k,l$ are cardinality terms then $k=l$ replaces $k\leq l$ and $k\geq l$ 
\item if $s,t$ are set terms then $s=t$ replaces $s\subseteq t$ and $s\supseteq t$
\end{itemize}
\begin{mydef}[Induced Interpretation $\mathcal{I}_I$]
An interpretation $\mathcal{I}_I$ can be induced from a constraint set $S$ by the following steps:
\begin{itemize}
\item for each variable $x\in Var(S)$ we introduce $x^{\mathcal{I}_I}$ and add it to $\Delta^{\mathcal{I}_I}$
\item for each $x:C$ such that $C$ is a concept name we add $x^{\mathcal{I}_I}$ to $C^{\mathcal{I}_I}$
\item for each $(x,y):C$ such that $C$ is a concept name we add $y^{\mathcal{I}_I}$ to $C^{\mathcal{I}_I}$
\item for each $(x,y):r$ such that $r$ is a role name we add $(x^{\mathcal{I}_I},y^{\mathcal{I}_I})$ to $r^{\mathcal{I}_I}$
\end{itemize}
\end{mydef}
\begin{mydef}[Tableau]
Let $S$ be a set of constraints and $\mathcal{I}_I$ be the induced interpretation from $S$
\begin{enumerate}
\item\label{cap} $\sqcap$-rule: $S$ contains $x:C_1\sqcap C_2$ but not both $x:C_1$ and $x:C_2$\\
$\rightarrow$ $S:=S\cup\{x:C_1, x:C_2\}$
\item\label{cup} $\sqcup$-rule: $S$ contains $x:C_1\sqcup C_2$ but neither $x:C_1$ nor $x:C_2$\\
$\rightarrow$ $S:=S\cup\{x:C_1\}$ or $S:=S\cup\{x:C_2\}$
\item\label{choose}$choose$-rule: $S$ contains
$x:succ(k\leq l$) and $(x,y):k^\prime$ with $k^\prime$ in $k$ but $(x,y):s\not\in S$ for some $|s|$ occurring in $k$\\
$\rightarrow$ for all $|s|$, in which $r$ occurs, either $S:=S\cup\{(x,y):s\}$ or $S:=S\cup\{(x,y):s^\neg\}$ and then jump to rule \ref{repeat}
\item\label{c}$cardinality$-rule: $S$ contains $x:succ(c)$, with $c\in\{k\leq l,\,k<l,\, n\, dvd\,l\}$, such that $c$ is violated in $\mathcal{I}_I$ regarding $x$
\begin{enumerate}
\item \label{setterm} if there is a positive set term $s$ in $l$\\
$\rightarrow$ introduce new variable $y$ and $S:=S\cup\{(x,y):s\}$, then jump to rule \ref{repeat}
\item \label{exceeded} if $l\in \mathbb{N}$ does not contain a set term then merge two successor $y_1\neq y_2$ of $x$ such that for a $|s|$ in $k$ we have $(x,y_1):s\in S$ and $(x,y_2):s\in S$ if no other constraints regarding $x$ become violated
\end{enumerate}
\item\label{s}$set$-rule: $S$ contains $x:succ(s_1\subseteq s_2)$ and $(x,y):s_1$ but not $(x,y):s_2$\\
$\rightarrow$ $S:=S\cup\{(x,y):s_2\}$ and then jump to rule \ref{repeat}
\item\label{repeat} $set.term$-rule (Repeat until inapplicable): In $S$ is $(x,y):s$ and
\begin{enumerate}
\item\label{setterm1} $s=s_1\cap s_2$ but $\{(x,y):s_1,\,(x,y):s_2\}\not\subseteq S$, then\\
$\rightarrow$ $S:=S\cup \{(x,y):s_1,\,(x,y):s_2\}$ 
\item\label{setterm2} $s=s_1\cup s_2$ and neither $\{(x,y):s_1\}\subseteq S$ nor $S\{(x,y):s_2\}\subset S$, then\\
$\rightarrow$ either $S:=S\cup \{(x,y):s_1\}$ or $S:=S\cup \{(x,y):s_2\}$ 
\item\label{setterm3} $s=C$ and $y:C\notin S$, where $C$ is an $\mathcal{ALCSCC}$ concepts, then \\
$\rightarrow$ $S:=S\cup\{y:C\}$
\end{enumerate}
\end{enumerate}
\end{mydef}
Note that:
\begin{itemize}
\item $\ref{exceeded}$ is never applicable for $n\, dvd\, l$
\item if  $n_1\,dvd\,n_2\cdot l$ and $mod(n_2,n_1)\neq 0$ then $n_1\,dvd\,l$ eventually
\end{itemize}
\begin{mydef}[Clash]
A constraint set $S$ contains a \textit{clash} if
\begin{itemize}
\item $\{x:\perp\}\subseteq S$ or
\item $\{x:A,\,x:\neg A\}\subseteq S$ or
\item $\{x:succ(c)\}\subseteq S$ and $c$ is violated regarding $x$
\end{itemize}
and no more rules are applicable.
\end{mydef}
\begin{mydef}[Derived Set]
A \textit{derived set} is a constraint set $S^\prime$ without clashes where only rule \ref{repeat} is not applicable.
\end{mydef}
The first rule decompose the conjunction and the second rule adds non-deterministically the right constraint. The third rule is important because we need to know of every successor what kind of role successors they are and in which concepts they are. We use $n_\mathcal{I}(x_{S_k})$ to count the successors of $x$ in $k$ which is important for detecting and avoiding violations of constraints. Now there might be a successor $y$ which satisfies only some part of $k$ in the given $S$ such that $n_\mathcal{I}(x_{S_k})$ does not count $y$. However there might be an interpretation $\mathcal{I^\prime}$ and the assignment $\pi_\mathcal{I^\prime}$ such that $n_{\mathcal{I}^\prime}(x_{S_k})$ $(x,y)$ counts $y$.\textcolor{red}{todo....}\\
We consider now two examples to explain the rule. \ref{exceeded}
\begin{ex}
Consider the following example 
\begin{align*}
S=\{x:succ(|r|=1)\sqcap succ(|r\cap s|=1)\sqcap succ=(|r\cap C|=1|)\\
x:succ(|r|=1), x:succ(|r\cap s|=1), x:succ=(|r\cap C|=1|)
\}
\end{align*}
If we try to satisfy at least two of the new constraints by the Tableau-algorithm above we end up with at least one constraint being violated. We use the rules on the three new constraints sequentially. Then we have 
\begin{figure}[H]
\centering
\begin{tikzpicture}
\node (0) at (0,0) {$x$};
\node (1) at (-4,-2) {$y_1$};
\node (2) at (0,-2) {$y_2$};
\node (3) at (4,-2) {$y_3$};
\node[below = 1mm of 3] {$C$};
\path[->] (0) edge node[above]{$r$} (1);
\path[->] (0) edge node[left]{$r\cap s$} (2);
\path[->] (0) edge node[above]{$r$} (3);
\end{tikzpicture}
\end{figure}
After using rule \ref{exceeded} two times we have the variable $x$ and its only $r\cap s$-successor $y$ which is of the concept $C$. We could use this rule because we do not violate any other constraints. 
\end{ex}
This condition helps to prevent an infinite chain of rule application:
\begin{ex}
\begin{align*}
S=\{x:succ(|r|<2)\sqcap succ(|r|\geq 2)\}	
\end{align*}
First we apply the rules \ref{setterm} and \ref{setterm3} two times to add two $r$-successors for $x$ hence $x:succ(|r|<2)$ is not satisfied any more. If we ignore the condition in rule \ref{exceeded} and apply it then we merge the two successors leading to $x:succ(|r|<2)$ being satisfied but $x:succ(|r|\geq 2)$ being violated. Then we apply the rules \ref{setterm} and \ref{setterm3} again leading to $x:succ(|r|<2)$ being violated again and so on. By the condition in \ref{exceeded} we can not use the rule which means the algorithm terminates with a clash stating that the constraint set is unsatisfiable.
\end{ex}
We also know that the application of rule \ref{repeat} eventually terminates because the number of concept names and role names are finite in $S$ (since $S$ is finite).\\ 
First we prove that the tableau algorithm terminates. 
\begin{mypro}
Let $C$ be a concept in negation normal form. Then there is no infinite chain of applications of any tableau rules issuing from $\{x:C\}$. 
\end{mypro}
To prove this we map any derived set $S$ to an element $\Psi(S)$ from a set $Q$. We then show that the elements in $Q$ can be ordered by a well-founded relation $\prec$. A well-founded relation says that there is no infinite decreasing chain. If we can show that by obtaining a derived set $S^\prime$ from another set $S$ we have $\Psi(S^\prime)\prec\Psi(S)$ then the algorithm terminates.\\
The elements in $Q$ are finite multisets of quintuples and the elements of the quintuples are either integers or mutlisets of integers. For two quintuples $q=(q_1,\dots,q_6)$ and $q^\prime=(q^\prime_1,\dots,q^\prime_6)$ it holds $q\prec q^\prime$ if for the first $i,\, 1\leq i\leq 6$, for which $q_i$ and $q_i^\prime$ differs it holds that $q_i\prec q_i^\prime$ (also called lexicographical ordering). For two mutlisets of integers $q_i$ and $q_i^\prime$ it holds $q_i^\prime\prec q_i$ if $q_i^\prime$ can be obtained from $q_i$ by replacing an integer $c$ in $q_i^\prime$ by a finite number of integers which are all smaller than $c$. The relation $\prec$ for those multisets is also well-founded because we work with integers. That means from a multiset $\{0,\,\dots\,,0\}$ we can not obtain a smaller multiset because we would have to replace at least one $0$ with integers which are smaller.\\
For a concept $C$ its size $size(C)$ is inductively defined as
\begin{itemize}
\item $0$, if $C$ is $\perp$
\item $1$, if $C$ is a primitive concept of $\mathbf{C}$
\item $size(\neg C)=size(C)$
\item $size(succ(c))= 1 + \sum_{C\in\mathbf{C}\text{ occurs in c}} size(C)$
\item $size(C\sqcap D)=size(C\sqcup D)=size(C)+size(D)$
\end{itemize}
The number $n_{sc}(x)$ denotes the number of constraints of the form $x:succ(c_1\subseteq c_2)$ for a variable $x$. Let $y$ be a successor of $x$. The number $n_{sc}(x,y)$ denotes the number of set constraints of the form $x:succ(c_1\subseteq c_2)$ where $(x,y):c_1\in S$ and $(x,y):c_2\in S$ hold.\\
The quintuples in $Q$ are defined as follows
\begin{mydef}
Let $S$ be a constraint set. The multiset $\Psi(S)$ consist of quintuples $\psi_S(x)$ for each variable $x$. The component of the quintuples are structured as follows
\begin{itemize}
\item the first component is a non-negative integer $max\{size(C)\mid x:C\in S\}$
\item the second component is a multiset of integers containing for each $x:C\sqcap D$, on which the $\sqcap$-rule is applicable, the non-negative integer $size(C\sqcap D)$ (respectively for $C\sqcup D$)
\item the third component denotes the number of all successors of $x$ in $S$
\item the fourth component is a multiset of integers containing for each $x:succ(k\leq n)\in S$ the number of all successors $y$ of $x$ such that for a role $r$ we have $(x,y):r$, $r$ occurs in $k$ but for at least one $|s|$ in $k$ we have neither $(x,y):s\in S$ nor $(x,y):\neg s\in S$
\item the fifth component is a multiset of integers in which for each successor $y$ of $x$ we have $n_{sc}(x)-n_{sc}(x,y)$
\end{itemize}
\end{mydef}
\begin{mylem}
The following properties hold
\begin{enumerate}
\item For any concept $C$ we have $size(C)\geq size(NNF(\neg C))$
 \item Any variable $y$ in a derived set $S$ has at most one predecessor $x$ in $S$
\item If $(x,y):r\in S$ for a $r\in\mathbf{R}$ (and $y$ is a introduced variable) then 
\begin{align*}
max\{size(C)\mid x:C\in S\}>max\{size(D)\mid y:D \in S\}
\end{align*}
\end{enumerate}
\end{mylem}
\begin{proof}$ $\\
\vspace*{-5mm}
\begin{enumerate}
\item By induction over the number of applications to compute the negation normal form we have $size(C)=size(NNF(\neg C))$. Because $\neg succ(k\geq0)$ can be replace by $\perp$ which is $smaller$ than $\neg succ(k\geq 0)$, we have $size(C)\geq size(NNF(\neg C))$. This can be done because $\neg succ(k\leq 0)= succ(k<0)$ which is impossible to satisfy and therefore $\neg succ(k\leq 0)=\perp$.
\item If $y$ is a newly introduced variable, then it can only be introduced by exactly one variable $x$ which is $y$'s only predecessor. If two variables are merged together by rule \ref{exceeded} then both variables must have the same predecessor $x$ by the condition of that rule.
\item By the second fact we know that $x$ is the only predecessor of $y$. When $y$ is introduced by applying \ref{setterm} on a constraint $x:succ(k\lesseqgtr l)$ then we have $y:C$ for every concept $C$ occurring in $l$ (for $\neg C$ we have $y:\neg C$). We know that $size(succ(k\lesseqgtr l))$ is greater then $size(C)=:max\{size(D)\mid y:D\in S\}$ therefore lemma 1.3 holds. A new constraint $y:D$ can occur either because rule \ref{cap} or \ref{cup} are applicable on $y:C$ with $C=D\sqcap D^\prime$ or $C=D\sqcup D^\prime$, which neither raise $max\{size(D)\mid y:D \in S\}$, or because rule \ref{choose} is applicable but that also does not raise $max\{size(D)\mid y:D \in S\}$: If rule \ref{choose} is applicable on $x:succ(k\leq l)$ then for every added constraint $y:D$ the concept $D$ must occur in $k$ and therefore lemma 1.3 holds. If $y$ gets merged together with another variable $z$, then $y$ and $z$ must have the same predecessor which means that all concept sizes regarding $z$ are also smaller then $max\{size(C)\mid x:C\in S\}$. 
\end{enumerate}
\end{proof}
From the next Lemma we can conclude that the Tableau-algorithm terminates.
\begin{mylem}
If $S^\prime$ is a derived set obtained from the derived set $S$, then $\Psi(S^\prime)\prec\Psi(S)$
\end{mylem}
The following proof is sectioned by the definition of obtaining a derived set.
\begin{proof}$ $\\
\vspace*{-5mm}
\begin{enumerate}
\item $S^\prime$ is obtained by the application of rule \ref{cap} on $x:C\sqcap D$:\\
The first component remains the same because $size(C)<size(C\sqcup D)$ and $size(D)<size(C\sqcap D)$. The second component is reduced because rule \ref{cap} can not be applied on $x:C\sqcap D$ any more meaning that the corresponding entry in the multiset is removed. If $C$ (or $D$) happens to be a disjunction ($C^\prime\sqcup D^\prime$) or a conjunction ($C^\prime\sqcap D^\prime$) then the second component also becomes smaller because $size(C^\prime)$ and $size(D^\prime)$ are always smaller than the disjunction or conjunction of them and therefore also smaller than $size(C\sqcap D)$. Hence the entry for $size(C\sqcap D)$ can be replace by the smaller $size(C^\prime\sqcup D^\prime)$ or $size(C^\prime\sqcap D^\prime)$.\\
Consider now a tuple $\psi_S(y)$ such that $x\neq y$. $\psi_S(y)$ can only be affected if $x$ is a successor of $y$. The first and second component of $\psi_S(y)$ remain unaffected because both are independent from $x$. The third and fourth component also remain unchanged because the number of $y$'s successors does not change. The fifth also do not change because we do not add a constraint of the form $(y,x):s$. Hence despite $x$ is a successor of $y$ by that rule $\psi_S(y)$ does not change.\\
This means, we can obtain $\Psi(S^\prime)$ from $\Psi(S)$ by replacing $\psi_S(x)$ with the smaller tuple $\psi_{S^\prime}(x)$. 
\item $S^\prime$ is obtained by the application of rule \ref{cup} on $x:C\sqcup D$:\\
similar to above
\item $S^\prime$ is obtained by the application of rule \ref{exceeded} on $x:succ(k<l)$ or $x:succ(k\leq l)$:\\
The first and second component of $\psi_S(x)$ remain unchanged. The third component decreases because we have one successor less and therefore $\psi_{S^\prime}(x)$ is smaller than $\psi_S(x)$. The new tuple $\psi_{S^\prime}(y)$ is also smaller than $\psi_S(x)$ because $y$ has the same constraints of the two merged successors whose first component are always smaller than the first component of $\psi_S(x)$ because of lemma 1.\\
No other tuples $\psi_S(z)$ are affected because otherwise $z$ must be a predecessor of $y$ and by Lemma 1 $z=x$. therefore $\Psi(S^\prime)$ can be obtained from $\Psi(S)$ by deleting the tuples of the two merged successors and by replacing $\psi(x)$ with the smaller tuples $\psi_{S^\prime}(x)$ and $\psi_{S^\prime}(y)$.
\item $S^\prime$ is obtained by the application of rule \ref{choose} on $x:succ(k<l)$ for a successor $y$ and of rule \ref{s}\\
After rule \ref{choose} we have either $(x,y):s$ or $(x,y):s^\neg$ for all $|s|$ in $k$. If all added constraints are negated the first four component of $\psi_S(x)$ remains unchanged but the fifth component decreases. We look now at an arbitrary positive constraint $(x,y):s$: We have to apply the rule \ref{repeat} until it is inapplicable. We know that at some point no more rules are inapplicable because $\mathbf{C}$ and $\mathbf{R}$ are finite and the constraints are in $NNF$. After rule \ref{repeat} we can have multiple $(x,y):r$, $r\in\mathbf{R}$, and/or $y:C$. For any $(x,y):r$ the third component of $\psi_S(x)$ might be decreased but never increased. For any $y:C$ $\psi_S(x)$ remains unchanged but we know that the first component of $\psi_S^\prime(y)$ is smaller then the first component of $\psi_S(x)$. In both cases the fifth component of $\psi_S(x)$ decreases by 1 because for each $|s|$ in $k$ we have either $(x,y):s$ or $(x,y):s^\neg$. \\ 
For any variable $z$ such that $z\neq y$. The tuple $\psi_S(z)$ is unaffected. It can only be affected by the rules if $z$ is a predecessor of $y$. But by lemma 1 that would mean that $z=x$.\\
We can obtained $\Psi(S^\prime)$ from $\Psi(S)$ by deleting $\psi_S(y)$ and replacing $\psi_S(x)$ by the two smaller sextuples $\psi_{S^\prime}(x)$ and $\psi_{S^\prime}(y)$.
\item $S^\prime$ is obtained by the application of rule \ref{setterm} on $x:succ(k<l)$, $x:succ(k\leq l)$ or $x:succ(n\,dvd\,l)$ and rule \ref{repeat}:\\ 
For a positive $s$ which occurs as $|s|$ in $l$ we introduce a new variable $y$ and add $(x,y):s$. The first two component of $\psi_S(x)$ remains unchanged. The third component of $\psi_S(x)$ must become smaller. It can not remain unchanged because this would mean that $n_l(x_S)\oset[-1ex]{.}{-}n_l(x_S)=0$ which denotes that we have already at least as many successor in $k$ as in $l$ and therefore would not have been able to apply this rule.\\
In $S^\prime$ exists now a new tuple $\psi_{S^\prime}(y)$. But since it was introduced by the constraint $x:succ(c)$, $c\in\{k<l,k\leq l,n\,dvd\,l\}$, the first component of it is always smaller then the first component of $\psi_S(x)$.\\
For any variable $z$ such that $z\neq y$. The tuple $\psi_S(z)$ is unaffected. It can only be affected by the rules if $z$ is a predecessor of $y$. But by lemma 1 that would mean that $z=x$.\\
Altogether $\Psi(S^\prime)$ can be obtained from $\Psi(S)$ by replacing $\psi_S(x)$ with the two smaller tuples $\psi_{S^\prime}(x)$ and $\psi_{S^\prime}(y)$.
\item $S^\prime$ is obtained by the application of rule \ref{s} on $x:succ(c_1\subseteq c_2)$ and rule \ref{repeat}:\\ 
After rule \ref{s} $S$ contains $(x,y):c_2$. Then rule \ref{repeat} is applied until inapplicable. Again this chain terminated eventually. After rule \ref{repeat} we can have multiple $(x,y):r$, $r\in\mathbf{R}$, and/or $y:C$. For any $(x,y):r$ the third component of $\psi_S(x)$ might be decreased but never increased. For any $y:C$ $\psi_S(x)$ remains unchanged but we know that the first component of $\psi_S^\prime(y)$ is smaller then the first component of $\psi_S(x)$. In the end the sixth component always decreases. We also know that the first component of $\psi_{S^\prime}(y)$ is always smaller than the first component of $\psi_S(x)$ by the definition of $|succ(c)|$.\\
For any variable $z$ such that $z\neq y$. The tuple $\psi_S(z)$ is unaffected. It can only be affected by the rules if $z$ is a predecessor of $y$. But by lemma 1 that would mean that $z=x$.\\
Therefore $\Psi(S^\prime)$ can be obtained from $\Psi(S)$ by deleting $\psi_S(y)$ and by replacing $\psi_S(x)$ with the two smaller tuples $\psi_{S^\prime}(x)$ and $\psi_{S^\prime}(y)$.
\end{enumerate}
\end{proof}
\end{document}