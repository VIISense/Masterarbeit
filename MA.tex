\documentclass[a4paper,11pt]{scrartcl}
\usepackage[english]{babel}  % falls der Artikel auf Deutsch verfasst ist
% Verwenden Sie nur EINE der beiden folgenden Zeilen, je nachdem, ob Ihr
% Betriebssystem LATIN1 (=ISO-8859-1) oder UTF8 als Zeichenkodierung
% verwendet. Ob Sie die richtige verwenden, merken Sie daran, dass
% die Umlaute richtig im Dokument dargestellt werden.
\usepackage[utf8]{inputenc}
\usepackage{amsmath}
\usepackage{ulem}
\usepackage{amssymb}
\usepackage{mathtools}
\usepackage{amsthm}
\usepackage{stmaryrd}
\usepackage{dsfont}
\usepackage{float}
\newtheoremstyle{break}
  {\topsep}{\topsep}%
  {\itshape}{}%
  {\bfseries}{}%
  {\newline}{}%
\theoremstyle{break}
\theoremstyle{definition}
\newtheorem{mydef}{Definition}
\newtheorem{mylem}{Lemma}
\newtheorem{mythe}{Theorem}
\newtheorem{mycol}{Corollary}
\newtheorem{mypro}{Proposition}
\newtheorem{ex}{Example}
\usepackage{color}
\usepackage{paralist}
\usepackage{tikz}
\usetikzlibrary{shapes,snakes,arrows}
\usepackage{verbatim}
\usetikzlibrary{positioning}
\usepackage{algpseudocode}
\usepackage{algorithm}
\algnewcommand\algorithmicforeach{\textbf{for each}}
\algdef{S}[FOR]{ForEach}[1]{\algorithmicforeach\ #1\ \algorithmicdo}
\usepackage{ulem}
\newcommand{\RM}[1]{\MakeUppercase{\romannumeral #1{}}}
\usepackage{url}
\makeatletter
\newcommand{\oset}[3][0ex]{%
  \mathrel{\mathop{#3}\limits^{
    \vbox to#1{\kern-2\ex@
    \hbox{$\scriptstyle#2$}\vss}}}}
\makeatother
\begin{document}
\section{Introduction}
Traditional data bases where data are stored solely without any connection towards to themselves like many people would imagine are often not enough any more. In biological and (bio)medical researches data bases are often based on ontologies \cite{bio}. Ontologies (in the computer science field) can be viewed as formal representation of a certain domain of interest. In data base they are collection of relation between the entities in the data base and are formulated as a fragment of first-order logic (FOL). These fragments of FOL are represented as \textit{Description Logic (DL)}, which is a family of knowledge representation system. DL are mainly built of concepts, which correspond to unary relations in FOL, and relation between the concepts, which correspond to binary relations in FOL. For more complex (compound) concepts operators like $\sqcap$, $\sqcup$,$\sqsubseteq$, $\exists$ and $\forall$, depending on the DL, are used. For example the statement "All Men and Women are Human" is formalize in DL as an \textit{axiom} $Men\,\sqcup\, Women\sqsubseteq Human$ and in FOL as $\forall x.Men(x)\vee Women(x)\rightarrow Human(x)$. The statement "All Human, who has children, are parents" in DL can be formalized a $Human\sqcap \exists hasChildren.\top \sqsubseteq Parent$ and in FOL as $\forall x \exists y. Human(x)\wedge hasChildren(x,y)\rightarrow Parent(x)$. Restriction with the operators $\exists$ and $\forall$ are called \textit{quantified} restrictions. The second statement can also be formalized with a \textit{qualified} restriction: $Human\sqcap \geq 1 hasChildren.\top\sqsubseteq Parent$. Each quantified restriction can be transformed into a qualified restriction.\\
One big research field in DL is \textit{Reasoning} which is the investigation of whether certain information can be concluded from the current data or not.\\
\textcolor{red}{to be continued...}
\section{Preliminaries}
In this work $\mathbf{C}$ denotes a set of concept names and $\mathbf{R}$ a set of role names, which are disjoint.
\begin{mydef}[$QFBAPA$]
Let $T$ be a set of symbols
\begin{itemize}
\item set terms over $T$ are:
\begin{itemize}
\item empty set $\emptyset$ and universal set $
\mathcal{U}$
\item every set symbol in $T$
\item if $s,t$ are set terms then also $s\cap t$, $s\cup t$ and $s^{\neg}$
\end{itemize}
\item set constraints over $T$ are
\begin{itemize}
\item $s\subseteq t$ and $s\not\subseteq t$
\item $s=t$ and $s\neq t$
\end{itemize}
where $s,t$ are set terms
\item cardinality terms over $T$ are:
\begin{itemize}
\item every number $n\in \mathbb{N}$
\item $|s|$ if $s$ is a set term
\item if $k,l$ are cardinality terms then also $k+l$ and $n\cdot k$, $n\in \mathbb{N}$
\end{itemize}
\item cardinality constraints over $T$ are:
\begin{itemize}
\item $k=l$ and $k\neq l$
\item $k<l$ and $k\geq l$
\item $k\leq l$ and $k>l$
\item $n$ $dvd$ $k$ and $n$ $\neg dvd$ $k$
\end{itemize}
where $k,l$ are cardinality terms and $n\in\mathbb{N}$
\end{itemize}
\end{mydef}
For readability we use $\lesseqgtr$ to address the comparison symbols $=,\,\leq,\,\geq,\,<,\,>$. The negation $\not\lesseqgtr$ address the symbols $\neq,\,>,\,<,\,\geq,\,\leq$ respectively.\\
Since $s\subseteq t$ can be expressed as the cardinality constraint $|s\cap t^\neg|\leq 0$ we will not consider any set constraints further in this work. In case we want to express $x:succ(s=t)$, with $s,t$ being set terms, we write instead $x:succ(|s\cap t^\neg|\leq 0)\sqcap succ(|s^\neg\cap t|\leq 0)$.
\begin{mydef}[$\mathcal{ALCSCC}$]
$\mathcal{ALCSCC}$ concepts are defined inductively:
\begin{itemize}
\item all concept names
\item $succ(c)$ if $c$ is a cardinality constraint over $\mathcal{ALCSCC}$ concepts and role names
\item if $C,D$ are concepts then:
\begin{itemize}
\item $\neg C$
\item $C\sqcup D$
\item $C\sqcap D$
\end{itemize}
\end{itemize}
\end{mydef}
An ABox $S$ in $\mathcal{ALCSCC}$ is a finite set of assertions of the form $x:C$ and $(x,y):s$, where $C$ is a $\mathcal{ALSCSS}$ concept, $s$ a set term and $x,y$ variables. The set $Var(S)$ is the set of variables occurring in $S$. 
\begin{mydef}[Interpretation]
An \textit{interpretation} $\mathcal{I}=(\Delta^\mathcal{I},\cdot^\mathcal{I},\pi_\mathcal{I})$ over an ABox $S$ in $\mathcal{ALCSCC}$ consists of a non-empty set $\Delta^\mathcal{I}$, an assignment $\pi_\mathcal{I}$ and a mapping $\cdot^\mathcal{I}$ which maps:
\begin{itemize}
\item $\emptyset$ to $\emptyset^\mathcal{I}$
\item $\mathcal{U}$ to $\mathcal{U}^\mathcal{I}\subseteq \Delta^\mathcal{I}$
\item each variable $x\in Var(S)$ to $x^\mathcal{I}\in \Delta^\mathcal{I}$
\item every concept names $A\in\mathbf{C}$ to $A^\mathcal{I}\subseteq \Delta^\mathcal{I}$
\item every role name $r\in\mathbf{R}$ to $r^\mathcal{I}\subseteq\Delta^\mathcal{I}\times\Delta^\mathcal{I}$, such that every element in $\Delta^\mathcal{I}$ has a finite number of successors.
\end{itemize}
The set $r^\mathcal{I}(x)$ contains all elements $y$ such that $(x,y)\in r^\mathcal{I}$ e.g. it contains all $r$-successors of $x$.\\
For compound concepts the mapping $\cdot^\mathcal{I}$ is extended inductively as follows
\begin{itemize}
\item $\top^\mathcal{I}=\Delta^\mathcal{I}$ and $\perp^\mathcal{I}=\emptyset^\mathcal{I}$
\item $(C\sqcap D)^\mathcal{I}:=C^\mathcal{I}\cap D^\mathcal{I}$, $(C\sqcup D)^\mathcal{I}:=C^\mathcal{I}\cup D^\mathcal{I}$
\item $(\neg C)^\mathcal{I}:=\Delta^\mathcal{I}\backslash C^\mathcal{I}$
\item $(s\cap t)^\mathcal{I}:= s^\mathcal{I}\cap t^\mathcal{I}$, $(s\cup t)^\mathcal{I}:= s^\mathcal{I}\cup t^\mathcal{I}$
\item $(s^\neg)^\mathcal{I}:=\mathcal{U}^\mathcal{I}\backslash s^\mathcal{I}$
\item $|s|^\mathcal{I}:=|s^\mathcal{I}|$
\item $(k+l)^\mathcal{I}:=(k^\mathcal{I}+l^\mathcal{I})$, $(n\cdot k)^\mathcal{I}:= n\cdot k^\mathcal{I}$
\item $succ(c)^\mathcal{I}=\{x\in \Delta^\mathcal{I}|$the mapping $\cdot^{\mathcal{I}_x}$ satisfies $c\}$
\end{itemize}
The mapping $\cdot^{\mathcal{I}_x}$ maps $\emptyset$ to $\emptyset^\mathcal{I}$, $\mathcal{U}$ to $\mathcal{U}^{\mathcal{I}_x}:=\{\bigcup_{r\in\mathbf{R}}r^\mathcal{I}(x)\}$, every concept $C$ occurring in $c$ to $C^{\mathcal{I}_x}:=C^\mathcal{I}\cap \mathcal{U}^{\mathcal{I}_x}$ and every role name $r$ occurring in $c$ to $r^{\mathcal{I}_x}:=r^\mathcal{I}(x)$.\\
The mappings satisfies for the cardinality terms $k,l$
\begin{itemize}
\item $k\lesseqgtr l$ iff $k^\mathcal{I}\lesseqgtr l^\mathcal{I}$
\item $n\,dvd\,l$ iff $\exists m\in\mathbb{N}:n\cdot m = l^\mathcal{I}$
\end{itemize}
The \textit{assignment} $\pi_\mathcal{I}:Var(S)\rightarrow\Delta^\mathcal{I}$ satisfies
\begin{itemize}
\item $x:C$ iff $\pi_\mathcal{I}(x)\in C^\mathcal{I}$ 
\item $(x,y):s$ iff $(\pi_\mathcal{I}(x),\pi_\mathcal{I}(y))\in s^\mathcal{I}$
\end{itemize} 
$\pi_\mathcal{I}$ satisfies an ABox $S$ if $\pi_\mathcal{I}$ satisfies every assertion in $S$. If $\pi_\mathcal{I}$ satisfies $S$ then $\mathcal{I}$ is a model of $S$.
\end{mydef}
\section{Tableau}
A Tableau-algorithm consist of completion rules to decide satisfiability of a set of assertions. The rules are applied exhaustively on the set until none is applicable any more. One major characteristic of this algorithm is that it does not matter in which order the rules are applied. Another characteristic is that it works non-deterministically: In case we have disjunctions we can choose between the concepts in this disjunctions. If a choice ends in a \textit{clash} then we track back to the point where we had to chose and take the other choice instead. If all choices ends in a clash then the ABox is unsatisfiable, otherwise it is satisfiable.\\
To maintain readability in this work we write $k\leq l$ instead of $l\geq k$ and $k<l$ instead of $l>k$. Therefore $k\lesseqgtr l$ can only represent $k\leq l$, $k=l$ or $k<l$ from now on.\\
To help the algorithm we want to avoid nested negation e.g. $\neg(\neg(\neg(A\cup B)))$. Hence we consider all concepts in \textit{negated normal form (NNF)}.
\begin{mydef}[Negation Normal Form]
A $\mathcal{ALCSCC}$ concept is in \textit{negation normal form} ($NNF$) if the negation sign $\neg$ appears only in front of a concept name or above a role name. Let $C$ be a arbitrary $\mathcal{ALCSCC}$ concept. With $NNF(C)$ we denote the concept which is obtained by applying the rules below on $C$ until none is applicable any more.
\begin{figure}[H]
\begin{minipage}[t]{.5\textwidth}
\raggedright
\begin{itemize}
\item $\neg\top$ $\rightarrow$ $\perp$
\item $\neg\perp$ $\rightarrow$ $\top$
\item $\neg\neg C$ $\rightarrow$ $C$
\item $\neg(C\sqcap D)$ $\rightarrow$ $\neg C \sqcup \neg D$
\item $\neg(C\sqcup D)$ $\rightarrow$ $\neg C \sqcap \neg D$
\item $C^\neg$ $\rightarrow$ $\neg C$
\item $\neg succ(c)$ $\rightarrow$ $succ(\neg c)$
\end{itemize}
\end{minipage}% <---------------- Note the use of "%"
\begin{minipage}[t]{.5\textwidth}
\raggedleft
\begin{itemize}
\item $\neg (k\lesseqgtr l)$ $\rightarrow$ $k\not\lesseqgtr l$
\item $\neg (k\not\lesseqgtr l)$ $\rightarrow$ $k\lesseqgtr l$
\item $\neg (n\text{ } dvd \text{ } k)$ $\rightarrow$ $n\text{ } \neg dvd \text{ } k$
\item $\neg (n\text{ } \neg dvd \text{ } k)$ $\rightarrow$ $n\text{ } dvd \text{ } k$
\item $(s\cap t)^\neg$ $\rightarrow$ $s^\neg \cup t^\neg$
\item $(s\cup t)^\neg$ $\rightarrow$ $s^\neg \cap t^\neg$
\item $(s^\neg)^\neg$ $\rightarrow$ $s$
\end{itemize}
\end{minipage}
\end{figure}
\end{mydef}
The rule $C^\neg\rightarrow \neg C$ is necessary because $C^\neg$ can be a result of $s^\neg$, where $s$ is a set term. It can be transformed into $\neg C$: For every interpretation $\mathcal{I}$ of $S$ we have $(C^\neg)^\mathcal{I}=\mathcal{U}\backslash C^\mathcal{I}$ and $(\neg C)^\mathcal{I}=\Delta^\mathcal{I}\backslash C^\mathcal{I}$. Since $\mathcal{U}\subseteq \Delta$ we can conclude that every element in $(C^\neg)^\mathcal{I}$ is also in $(\neg C)^\mathcal{I}$.\\
Next we introduce \textit{induced interpretation} with which we can count successors of variables after any rule application.
\begin{mydef}[Induced Interpretation]
An interpretation $\mathcal{I}(S)$ can be induced from an ABox $S$ by the following steps:
\begin{itemize}
\item for each variable $x\in Var(S)$ we introduce $x^{\mathcal{I}(S)}$ and add it to $\Delta^{\mathcal{I}(S)}$
\item for each $x:C$ such that $C$ is a concept name we add $x^{\mathcal{I}(S)}$ to $C^{\mathcal{I}(S)}$
\item for each $(x,y):r$ such that $r$ is a role name we add $(x^{\mathcal{I}(S)},y^{\mathcal{I}(S)})$ to $r^{\mathcal{I}(S)}$
\end{itemize}
\end{mydef}
Since we can now denote the number of successor of a variable $x$ we can determine which assertion of the form $x:succ(c)$ are violated.
\begin{mydef}[Violated assertion]
Let $S$ be a set of assertion, $x$ be a variable, $k$ be a cardinality term and $n\in\mathbb{N}$. An assertion is \textit{violated} if
\begin{itemize}
\item $x:succ(k\lesseqgtr n)$ and $k^{\mathcal{I}(S)_x}\not\lesseqgtr n$
\item $x:succ(k\lesseqgtr l)$ and $k^{\mathcal{I}(S)_x}\not\lesseqgtr l^{\mathcal{I}(S)_x}$
\item $x:succ(n\,dvd\,k)$ and $mod(k^{\mathcal{I}(S)_x},n)\neq 0$
\end{itemize} 
where $n\in\mathbb{N}$.
\end{mydef}
Like already mentioned an ABox is unsatisfiable if all choices ends in a clash. A clash in a ABox $S$ implies that $\perp$ can be derived from $S$.
\begin{mydef}[Clash]
An ABox $S$ contains a \textit{clash} if
\begin{itemize}
\item $\{x:\perp\}\subseteq S$ or
\item $\{x:A,\,x:\neg A\}\subseteq S$ or
\item $\{(x,y):s,\,(x,y):s^\neg\}\subseteq S$ or
\item $\{x:succ(c)\}\subseteq S$ violated and no more rules are applicable
\end{itemize}
\end{mydef}
Also important for the algorithm is to consider the \textit{signs} of concept names and role names.
\begin{mydef}[Positive and Negative Sign]
Let $(x,y):s$ be an arbitrary assertion with $x,y\in Var(S)$ and $s$ being a set term in $NNF$. A concept name $C$ has a \textit{positive sign} in $s$ if no negation sign appears immediately in front of $C$. It has a \textit{negative sign} otherwise. A role name $r$ has a \textit{positive sign} if no negation sign appears above it. It has a \textit{negative sign} otherwise.
\end{mydef}
All concept names in $\mathbf{C}$ and role names in $\mathbf{R}$ have a positive sign.
\subsection{Restrictions}\label{restriction}
The Tableau-algorithm for $\mathcal{ALC}$ is, as expected, straightforward: We apply rules in a arbitrary order whenever they are applicable. However for DLs which are more expressive then $\mathcal{ALC}$ some caution have to be made. In \cite{1} the considered DL is $\mathcal{ALCQ}$ which additionally allows qualified number restrictions. The algorithm uses a rule for replacing variables and introduces a safeness definition to prevent endless loops of rule application, which can occur due to the replication. In \cite{Ba},\cite{2} and\cite{6} the considered DLs allow inverse roles which is handled with blocking techniques. The DL $\mathcal{ALCSCC}$ is a very expressive concept language and hence there are some difficulty to handle for the Tableau-algorithm.\\
We loose a bit the property of a Tableau-algorithm that rules can be applied in any order: In case we add $(x,y):s$ to our ABox and $s$ is a (always finite) chain of disjunction and conjunction we want to add all assertions of $y$ first before applying any other rules. This way $y$ has all assertion we want to assign to it and hence avoid adding unnecessary variables which can lead into a violation of other assertions.\\
Similar to \cite{1} and \cite{6}, where a variable can be replaced by another variable, we can merge two variables during the Tableau-algorithm like in \cite{2}.
\begin{mydef}[Merge]
\textit{Merging} $y_1$ and $y_2$ results in one variable $y$: replace all occurrence of $y_1$ and $y_2$ with $y$. 
\end{mydef}
For the Tableau-algorithm in this work a merging can only occur if an assertion $x:succ(k\lesseqgtr l)$ is violated. The merging is only reasonable if it reduces $k$. It can be reasonable if by merging $l$ increases, for example $x:succ(1\leq |r\cap t|)$ with $(x,y_1):r$ and $(x,y_2):t$. However the easiest solution is just to add an $r\cap t$-successor.\\
By merging variables assertion may become violated. To ensure the termination of the algorithm we intuitively want to avoid violating any assertions especially when they are satisfied. However there are cases where a violation is unavoidable. We look at the following example:
\begin{ex}
\begin{align*}
&S=\{x:succ(|t|+|A\cup B|\geq 4),\,x:succ(|A\cup B|\leq 1)\\
&\quad\quad y_1:A,\,y_2:B,\,(x,y_1):t,\,(x,y_2):t\}
\end{align*}
\end{ex}
We see that the assertion $x:succ(|A\cup B|\leq 1)$ is violated and a solution is to merge $y_1$ and $y_2$. This leads to $x:succ(|t|+|A\cup B|\geq 4)$ being violated. We can fix it by adding new successors. We can easily detect that adding a $t$-successor leads to a satisfied ABox. On the other hand a successor in $A\cup B$ leads to the initial state where $x:succ(|A\cup B|\leq 1)$ is violated. Hence the algorithm can run into a endless loop of adding and merging variables. Therefore we introduce a notion of \textit{blocking}. If we merged two variable $y_1$ and $y_2$ into $y$ because of a violated assertion $x:succ(k\lesseqgtr l)$, $k=n_0+n_1\cdot |s_1|+\dots + n_j\cdot|s_j|$, then we want to \textit{block} any introduction of an assertion of the form $(x,z):s_1\cap\dots \cap s_i$, $1\leq i\leq j$. This way we want to avoid a possible re-violation of $x:succ(k\lesseqgtr l)$ and possible unless loops of merging and introducing variables. For that we introduce for a variable $x$ a blocking set $b(S,x)$ in which set terms of $k$ are listed.
\begin{mydef}[Blocking]
Let $y_1$ and $y_2$ be successors of a variable $x$. If $y_1$ and $y_2$ are merged into $y$ because of an assertion $x:succ(k\lesseqgtr l)$ then $b(S,x):=b(S,x)\cup\{s_i|k=n_0+n_1\cdot|s_1|+\dots+n_j\cdot|n_j|,\,\forall i:1\leq i\leq j\}$. A set term $u=t_1\cap\dots\cap t_j$ is \textit{blocked} by $x$ in the ABox $S$ if $\forall i:1\leq i\leq j,\, t_i\in b(S,x)$.\\
At the beginning of the Tableau-algorithm all sets $b(S,x)$, $x\in Var(S)$, are empty.
\end{mydef}
In our example after the merging the set term $A\cup B$ is added to $b(S,x)$ and hence this set term is blocked by $x$, which means that to satisfy $x:succ(|t|+|A\cup B|\geq 4)$ we can only add a $t$-successor. 
Also like in \cite{1} and \cite{6} we have to be \textit{safe} when introducing new variables otherwise we may end in a endless loop or with a false output. In case of $x:succ(n\,dvd\,l)$ we do not have to consider any special cases because in worst case we have to add $n$ successors, which are counted in $l^{\mathcal{I}(S)_x}$. The same goes for assertions of the form $x:succ(k\lesseqgtr l)$, $k=n_0+n_1\cdot|s_1|+\dots+n_j\cdot|n_j|$ with $\forall i: 1\leq i\leq j,\, n_i=0$. Here again we just increase $l$ until the assertion is satisfied. However if $k$ and $l$ can both increase, we have to avoid cases where $k$ increases faster then $l$. The increment depends on the set term $u$, which is not blocked by $x$, for which we want to introduce a new variable $y$ and add $(x,y):u$ to $S$. To determine whether $u$ is \textit{safe} we count how often $u$ "appears" in $l$ and $k$. If it appears more often in $l$ than in $k$ then it is safe.
\begin{mydef}[Safe]
Let $x:succ(k\lesseqgtr l)$ be an assertion in $S$. Let $u=t_1\cap\dots \cap t_j$ with $1\leq i\leq j$. If $n_k(u)<n_l(u)$ and $u$ is not blocked by $x$ then $u$ is called \textit{safe regarding $k\leq l$}. The number $n_k(u)$ (and $n_l(u)$ respectively) is computed as followed:
\begin{algorithm}[H] \caption{Compute $n_k(u)$}
\begin{algorithmic}[l]
\State $n_k(u):=0$
\State $k=n_0+n_1\cdot|s_1|+\dots+n_j\cdot|s_j|$
\State $u=t_1\cap \dots \cap t_o$
\ForEach {$1\leq i\leq j:\, n_i\cdot|s_i|$, $s_i=s^\prime_1\cup \dots\cup s^\prime_p$, $p\in\mathbb{N}$}
\If {$\exists q, 1\leq q\leq p:$ $u=s^\prime_q\cap t^\prime$}
\State $n_k(u):=n_k(u)+n_i$
\EndIf
\EndFor\\
\Return $n_k(u)$
\end{algorithmic}
\end{algorithm}
\end{mydef}
This says that it is only safe to add a variable if $l$ increases faster then $k$.
As example we look at 
\begin{ex}
\begin{align*}
S=\{x:succ(|r\cup s|<|r|+|s|)\}
\end{align*}
\end{ex}
The set terms $r$ and $s$ are not blocked but still not safe because $n_{|r\cup s|}(r)=n_{|r|+|s|}(r)=1$ and $n_{|r\cup s|}(s)=n_{|r|+|s|}(s)=1$. However the set term $r\cap t$ is safe because $r\cap t\notin b(S,x)=\emptyset$ and $1=n_{|r\cup s|}(r\cap s)<n_{|r|+|s|}(r\cap s)=2$. 
\subsection{Algorithm}
For the Tableau-algorithm we define the properties of the following notations:
\begin{itemize}
\item Conjunction binds stronger than disjunction: $s\cup t\cap u = s\cup (t\cap u)$
\item if $k,l$ are cardinality terms then $k=l$ replaces $k\leq l$ and $k\geq l$ 
\end{itemize}
\begin{mydef}[Tableau]
Let $S$ be a set of assertions in simplified $NNF$.
\begin{enumerate}
\item\label{cap} $\sqcap$-rule: $S$ contains $x:C_1\sqcap C_2$ but not both $x:C_1$ and $x:C_2$\\
$\rightarrow$ $S:=S\cup\{x:C_1, x:C_2\}$
\item\label{cup} $\sqcup$-rule: $S$ contains $x:C_1\sqcup C_2$ but neither $x:C_1$ nor $x:C_2$\\
$\rightarrow$ $S:=S\cup\{x:C_1\}$ or $S:=S\cup\{x:C_2\}$
\item\label{choose}$choose$-rule: $S$ contains
\begin{itemize}
\item $x:succ(k\lesseqgtr l)$
\item $(x,y):k^\prime$, $k=n\cdot|k^\prime\cup u_1|+m\cdot|k^\prime\cap u_2|+u_3$, $n,m\in\mathbb{N}_0$, $u_1,u_2,u_3$ are set terms
\item but not $(x,y):k$
\end{itemize}
$\rightarrow$ either $S:=S\cup\{(x,y):k\}$ or $S:=S\cup\{(x,y):k^\neg\}$. Then jump to rule \ref{repeat}
\item\label{chooserole}$choose$-$a$-$role$-rule: $S$ contains $(x,y):s$ but for any $r\in\mathbf{R}$: $(x,y):r\notin S$\\
$\rightarrow$ choose $r\in\mathbf{R}$, such that $(x,y):r^\neg\notin S$. $S:=S\cup\{(x,y):r\}$. Then jump to rule \ref{repeat}
\item\label{dvd}$divide$-rule: $S$ contains $x:succ(n\,dvd\,l)$, $l=n_1\cdot|s_1|+\dots+n_i\cdot|s_i|+\dots+n_j\cdot|s_j|$, which is violated\\
$\rightarrow$ introduce a new variable $y$, choose $s=s_1\cap \dots \cap s_i$, $1\leq i\leq j$ and $S:=S\cup\{(x,y):s\}$. Then jump to rule \ref{repeat}
\item\label{leq}$\leq$-rule: $S$ contains 
\begin{itemize}
\item $x:succ(k\lesseqgtr l)$, which is violated
\item there set term $s:=|s_1\cap \dots \cap s_i|$, $l=n_1\cdot|s_1|+\dots+n_i\cdot|s_i|+\dots+n_j\cdot|s_j|$, which is safe regarding $k\lesseqgtr l$
\end{itemize}
$\rightarrow$ $S:=S\cup\{(x,y):s\}$. Then jump to rule \ref{repeat}
\item\label{exceeded}$merge$-rule: $S$ contains
\begin{itemize}
\item $x:succ(k\lesseqgtr l)$, which is violated
\item $(x,y_1):s_1$ and $(x,y_2):s_2$, such that $y_1\neq y_2$ and $k=n\cdot |s_1\cup s_2|+u$, where $u$ is a cardinality term
\end{itemize}
$\rightarrow$ merge $y_1$ and $y_2$ 
\item\label{s}$\leq 0$-rule: $S$ contains 
\begin{itemize}
\item $x:succ(|s_1\cap\dots\cap s_j|\leq 0)$
\item $(x,y):s_1$, $\dots$, $(x:y):s_i$, $1\leq i<j$
\item but not $(x,y):s_{i+1}$, $\dots$, $(x,y):s_j$
\end{itemize}
$\rightarrow$ choose $n\in\{i+1,\dots, j\}$, extend $S:=S\cup\{(x,y):s_n^\neg\}$ and then jump to rule \ref{repeat}
\item\label{repeat} $set.term$-rule (Repeat until inapplicable): In $S$ is $(x,y):s$ and
\begin{enumerate}
\item\label{setterm1} $s=s_1\cap s_2$ but $\{(x,y):s_1,\,(x,y):s_2\}\not\subseteq S$\\
$\rightarrow$ $S:=S\cup \{(x,y):s_1,\,(x,y):s_2\}$ 
\item\label{setterm2} $s=s_1\cup s_2$ and neither $\{(x,y):s_1\}\subseteq S$ nor $S\{(x,y):s_2\}\subset S$\\
$\rightarrow$ either $S:=S\cup \{(x,y):s_1\}$ or $S:=S\cup \{(x,y):s_2\}$ 
\item\label{setterm3} $s=C$ and $y:C\notin S$, where $C$ is an $\mathcal{ALCSCC}$ concepts\\
$\rightarrow$ $S:=S\cup\{y:C\}$
\end{enumerate}
\end{enumerate}
\end{mydef}
\begin{mydef}[Derived Set]
A \textit{derived set} is an ABox $S^\prime$ where rule \ref{repeat} is not applicable.
\end{mydef}
In order words a derived set is an ABox on which we applied a rule completely e.g. every time we add a new assertion $(x,y):s$ we add all assertion concluded by it to $S$ first.\\
We now explain the rules of the Tableau-algorithm and their intention, if not already mention in Section \ref{restriction}.\\
The first rule decompose the conjunction and the second rule adds non-deterministically the right assertion.\\
The $choose$-rule is important because we need to know of every successor what kind of role successors they are and in which concepts they are. For an assertion $x:succ(k\lesseqgtr l$ it is important that $k^{\mathcal{I}(S)_x}$ and $l^{\mathcal{I}(S)_x}$ counts the successors correctly. In the following case the successor $y$ is not counted in $l^{\mathcal{I}(S)_x}$ while $x:succ(k\leq l)$ is violated.
\begin{ex}
\begin{align*}
S=\{x:succ(1\leq|r\cap s|), (x,y):r\}
\end{align*}
\end{ex}
There might be an model $\mathcal{I^\prime}$ where $y$ is also a $s$-successor of $x$ and hence $l^{\mathcal{I}(S)_x}<l^{\mathcal{I}^\prime(S)_x}$. However the Tableau-algorithm should be able to construct every model of $S$ if $S$ is consistent. Therefore this rule adds non-deterministically either $(x,y):s$ or $(x,y):s^\neg$ which are the only two possibilities. This way we are also able to construct $\mathcal{I^\prime}$.\\
The $choose$-$a$-$role$-rule is necessary because for a assertion $x:succ(c)$ we might have no role name with a positive sign in $c$. Which means we know $x$ must have some successors but we can not decide which role-successor it is. As example we have
\begin{ex}
\begin{align*}
&\mathbf{R}=\{r,s\}\\
&S=\{x:succ(|r^\neg|\geq 1)\}
\end{align*} 
\end{ex}
It states that $x$ have at least one successor which is not a $r$-successor. Since $\mathbf{R}$ only contains $r$ and $s$ we know that the successors must be $s$-successors. First we apply rule \ref{leq} to actually add a successor. Therefore $y$ is introduced and $(x,y):r^\neg$ is added to $S$. Now no more rules are applicable except for the $choose$-$a$-$role$-rule. With that rule we can pick either $r$ or $s$. We can not pick $r$ because $r^\neg$ occurs in the assertion. Therefore we have to pick $s$. Another more simple but not so significant example is
\begin{ex}
\begin{align*}
&\mathbf{R}=\{r,s\}\\
&S=\{x:succ(|A|\geq 1)\}
\end{align*} 
\end{ex}
We know that $x$ must have a successor in $A$ but we still need to assign a role. In this case we can choose between $r$ and $s$.
\\
The $divide$-rule is straightforward: We choose one set term $s=s_1\cap\dots\cap s_i$ such that $l=n_1\cdot|s_1|+\dots+n_i\cdot|s_i|+\dots+n_j\cdot|s_j|$ and introduce a new variable $y$ and add $(x,y):s$ to $S$. For any $x:succ(n\,dvd\,l)$ we know that the chain of this rule application is finite because in worst case we have to introduce $n$ new variables with the same set term.
\\
The reason and idea of the $\leq$-rule is written in Section \ref{restriction}.\\
The same goes for the $merge$-rule.\\
The $\leq 0$-rule deal with an assertion with a set constraint $s_1\subseteq s_2$, which is written here as cardinality constraint $|s_1\cap s_2^\neg|\leq 0$. Those cardinality constraint can not be dealt with the other rules. In case the left side has at least three set term  e.g. $|s_1\cap s_2\cap s_3|$ we have can have multiple possible solutions e.g. $(x,y):s_1\cap s_2\cap s_3^\neg$, $(x,y):s_1\cap s_2^\neg\cap s_3$ and $(x,y):s_1\cap s_2^\neg\cap s_3^\neg$. Hence we let the algorithm choose and backtrack if needed.\\
The $set.term$-rules are applied immediately after a new assertions $(x,y):s$ is added to $S$. The reason for that is, that we want to add all needed assertions for $y$ and hence update all $k^{\mathcal{I}(S)_x}$ correctly. We know that the number of this application is finite because an ABox is finite and hence the number of concept names and role names occurring in this ABox is also finite. Since the constraints are in $NNF$ set terms can never be infinite and hence this rule applies only a finite times.
\iffalse
\section{Correctness}
For the correctness proof of the Tableau-algorithm we have to show that
\begin{itemize}
\item For every input the Tableau-algorithm terminates
\item If no more rules are applicable on a clash-free ABox $S$ then $S$ is satisfiable
\item If $S$ is satisfiable then the Tableau-algorithm terminates without a clash
\end{itemize}
First we prove that the tableau algorithm terminates. 
\begin{mypro}
Let $C$ be a concept in simplified $CNNF$. Then there is no infinite chain of applications of any tableau rules issuing from $\{x:C\}$. 
\end{mypro}
To prove this we map any derived set $S$ to an element $\Psi(S)$ from a set $Q$. We then show that the elements in $Q$ can be ordered by a well-founded relation $\prec$. A well-founded relation says that there is no infinite decreasing chain. If we can show that by obtaining a derived set $S^\prime$ from another set $S$ we have $\Psi(S^\prime)\prec\Psi(S)$ then the algorithm terminates.\\
The elements in $Q$ are finite multisets of septuples and the elements of the septuples are either integers or mutlisets of integers. For two septuples $q=(q_1,\dots,q_7)$ and $q^\prime=(q^\prime_1,\dots,q^\prime_7)$ it holds $q\prec q^\prime$ if for the first $i,\, 1\leq i\leq 7$, for which $q_i$ and $q_i^\prime$ differs it holds that $q_i\prec q_i^\prime$ (also called lexicographical ordering). For two mutlisets of integers $q_i$ and $q_i^\prime$ it holds $q_i^\prime\prec q_i$ if $q_i^\prime$ can be obtained from $q_i$ by replacing an integer $c$ in $q_i^\prime$ by a finite number of integers which are all smaller than $c$. The relation $\prec$ for those multisets is well-founded because we work with integers. That means from a multiset $\{0,\,\dots\,,0\}$ we can not obtain a smaller multiset because we would have to replace at least one $0$ with integers which are smaller.\\
For a concept $C$ its size $size(C)$ is inductively defined as
\begin{itemize}
\item $0$, if $C$ is $\perp$
\item $1$, if $C$ is a concept name of $\mathbf{C}$
\item $size(\neg C)= 1+size(C)$
\item $size(succ(c))= 1 + \sum_{C\in\mathbf{C}\text{ in c}} size(C)$
\item $size(C\sqcap D)=size(C\sqcup D)=size(C)+size(D)$
\end{itemize}
The number $n_{sc}(x)$ denotes the number of assertions of the form $x:succ(s_1\subseteq s_2)$ for a variable $x$. Let $y$ be a successor of $x$. The number $n_{sc}(x,y)$ denotes the number of set assertions of the form $x:succ(c_1\subseteq c_2)$ where $(x,y):s_1\in S$ and $(x,y):s_2\in S$ hold.\\
The asymmetrical difference of two numbers $n,m$ is denoted by 
\begin{equation*}
n\unlhd m \begin{cases}
n-m& \text{if } n> m\\
0 & \text{if } n\leq m
\end{cases}
\end{equation*}
The septuples in $Q$ are defined as follows
\begin{mydef}
Let $S$ be an ABox. The multiset $\Psi(S)$ consist of septuples $\psi_S(x)$ for each variable $x$. The component of the septuples are structured as follows
\begin{itemize}
\item the first component is a non-negative integer $max\{size(C)\mid x:C\in S\}$
\item the second component is a multiset of integers containing for each $x:C\sqcap D$, on which the $\sqcap$-rule is applicable, the non-negative integer $size(C\sqcap D)$ (respectively for $C\sqcup D$)
\item the third component is a multiset which denotes for every $x:succ(k\leq l)$ the integer $n(x,k,S)\unlhd n(x,l,S)$
\item the fourth component is a multiset of integers in which for each successor $y$ of $x$ we have $n_{sc}(x)-n_{sc}(x,y)$
\item the fifth component denotes the number of all successors of $x$ in $S$
\item the sixth component is a multiset of integers containing for each $x:succ(k\leq n)\in S$ the number of all successors $y$ of $x$ such that we have $(x,y):k^\prime$, $k^\prime$ occurs in $k$ but for at least one $|s|$ in $k$ we have neither $(x,y):s\in S$ nor $(x,y):\neg s\in S$
\item the seventh component saves the difference of the number of all successors and the number of successors $y$ for which there exists a positive role name $r$ such that $(x,y):r\in S$
\end{itemize}
\end{mydef}
\begin{mylem}
The following properties hold
\begin{enumerate}
\item For any concept $C$ we have $size(C)\geq size(NNF(\neg C))$
 \item Any variable $y$ in a derived set $S$ has at most one predecessor $x$ in $S$
\item If $(x,y):r\in S$ for a $r\in\mathbf{R}$ (and $y$ is a introduced variable) then 
\begin{align*}
max\{size(C)\mid x:C\in S\}>max\{size(D)\mid y:D \in S\}
\end{align*}
\end{enumerate}
\end{mylem}
\begin{proof}$ $\\
\vspace*{-5mm}
\begin{enumerate}
\item By induction over the number of applications to compute the negation normal form we have $size(C)=size(NNF(\neg C))$. Because $\neg succ(k\geq0)$ can be replace by $\perp$ which is $smaller$ than $\neg succ(k\geq 0)$, we have $size(C)\geq size(NNF(\neg C))$. This can be done because $\neg succ(k\geq 0)= succ(k<0)$ which is impossible to satisfy and therefore $\neg succ(k\geq 0)=\perp$.
\item If $y$ is a newly introduced variable, then it can only be introduced by exactly one variable $x$ which is $y$'s only predecessor. If two variables are merged together by rule \ref{exceeded} then both variables must have the same predecessor $x$ by the condition of that rule.
\item By the second fact we know that $x$ is the only predecessor of $y$. When $y$ is introduced by applying \ref{leq} on a assertion $x:succ(k\lesseqgtr l)$ then we have $y:C$ for every concept $C$ occurring in $l$ (for $\neg C$ we have $y:\neg C$). We know that $size(succ(k\lesseqgtr l))$ is greater then $size(C)=:max\{size(D)\mid y:D\in S\}$ therefore Lemma 1.3 holds. A new assertion $y:D$ can occur either because rule \ref{cap} or \ref{cup} are applicable on $y:C$ with $C=D\sqcap D^\prime$ or $C=D\sqcup D^\prime$, which neither raise $max\{size(D)\mid y:D \in S\}$, or because rule \ref{choose} is applicable but that also does not raise $max\{size(D)\mid y:D \in S\}$: If rule \ref{choose} is applicable on $x:succ(k\leq l)$ then for every added assertion $y:D$ the concept $D$ must occur in $k$ and therefore $size(succ(k\leq))>size(D)$. If $y$ gets merged together with another variable $z$, then $y$ and $z$ must have the same predecessor which means that all concept sizes regarding $z$ are also smaller then $max\{size(C)\mid x:C\in S\}$. 
\end{enumerate}
\end{proof}
From the next Lemma we can conclude that the Tableau-algorithm terminates.
\begin{mylem}
If $S^\prime$ is a derived set obtained from the derived set $S$, then $\Psi(S^\prime)\prec\Psi(S)$
\end{mylem}
\begin{proof}
The following proof is sectioned by the definition of obtaining a derived set.
\begin{enumerate}
\item $S^\prime$ is obtained by the application of rule \ref{cap} on $x:C\sqcap D$:\\
The first component remains the same because $size(C)<size(C\sqcup D)$ and $size(D)<size(C\sqcap D)$. The second component decreases because rule \ref{cap} can not be applied on $x:C\sqcap D$ any more meaning that the corresponding entry in the multiset is removed. If $C$ (or $D$) happens to be a disjunction ($C^\prime\sqcup D^\prime$) or a conjunction ($C^\prime\sqcap D^\prime$) then the second component also becomes smaller because $size(C^\prime)$ and $size(D^\prime)$ are always smaller than the disjunction or conjunction of them and therefore also smaller than $size(C\sqcap D)$. Hence the entry for $size(C\sqcap D)$ can be replace by the smaller $size(C^\prime\sqcup D^\prime)$ or $size(C^\prime\sqcap D^\prime)$.\\
Consider now a tuple $\psi_S(y)$ such that $x\neq y$. $\psi_S(y)$ can only be affected if $x$ is a successor of $y$. The first and second component of $\psi_S(y)$ remain unaffected because both are independent from $x$. The third component can decrease but never increase: By adding an assertion for $x$ the number $n_{sc}(y,x)$ might increases and hence the component also might decreases. The fourth, fifth and sixth component also remain unchanged because the number of $y$'s successors does not change. The sixth also do not change because we do not add an assertion of the form $(y,x):s$. Hence $\psi_S(y)$ does not change.\\
This means that we can obtain $\Psi(S^\prime)$ from $\Psi(S)$ by replacing $\psi_S(x)$ with the smaller tuple $\psi_{S^\prime}(x)$. 
\item $S^\prime$ is obtained by the application of rule \ref{cup} on $x:C\sqcup D$:\\
similar to above
\item $S^\prime$ is obtained by the application of rule \ref{choose} on $x:succ(k\leq l)$ for a successor $y$ and of rule \ref{s}\\
After rule \ref{choose} we have either $(x,y):s$ or $(x,y):s^\neg$ for all $|s|$ in $k$. Whether it is $(x,y):s$ or $(x,y):s^\neg$ the first two component do not change because we do not add any new assertions regarding $x$. The third and fifth component also does not change because we do not add any new successors for $x$. The fourth component might decreases but never increases: By adding assertions we can only increase the number $n_{sc}(x,y)$ which means that $n_{sc}(x)-n_{sc}(x,y)$ decreases. The sixth component of $\psi_S(x)$ decreases because $y$ does not hold the condition of the fifth component any more.  Hence $\psi_{S^\prime}\prec\psi_S(x)$.\\ 
For any variable $z$ such that $z\neq y$. The tuple $\psi_S(z)$ is unaffected. It can only be affected by the rules if $z$ is a predecessor of $y$. But by Lemma 1.2 that would mean that $z=x$.\\
Because $y$ is a successor of $x$ we know by Lemma 1.3 that the first component of $\psi_{S^\prime}(y)$ is smaller than the first component of $\psi_{S^\prime}(x)$ and therefore $\psi_{S^\prime}(y)\prec\psi_{S^\prime}(x)$. Since the first component of $\psi_{S^\prime}(x)$ does not change we also have $\psi_{S^\prime}(y)\prec\psi_{S}(x)$.\\
We can obtained $\Psi(S^\prime)$ from $\Psi(S)$ by deleting $\psi_S(y)$ and replacing $\psi_S(x)$ by the two smaller septuples $\psi_{S^\prime}(x)$ and $\psi_{S^\prime}(y)$.
\item $S^\prime$ is obtained by the application of rule \ref{chooserole} on $(x,y):s$:\\
The first and second component remains unchanged. Also the third and fifth component because we do not add new successors. The fourth component can decrease but never increase: By adding an assertion $(x,y):r, r\in\mathbf{R}$ we can only increase $n_{sc}(x,y)$ and therefore can only decrease the multiset. With a similar reasoning the sixth component can decrease but never increase. The seventh component always decreases because for a successor $y$ there was no positive role name $r$ such that $(x,y):r$ but after the rule application there is such an assertion. therefore the difference becomes smaller.\\
Let $z$ be a variable such that $z\neq y$. The element $\psi_S(z)$ can only change if $z$ is a predecessor of $y$. But by Lemma 1.2 that means that $z=x$.\\
We can obtained $\Psi(S^\prime)$ from $\Psi(S)$ by replacing $\psi_S(x)$ with the smaller $\psi_{S^\prime}(x)$.
\item $S^\prime$ is obtained by the application of rule \ref{leq} on $x:succ(k<l)$, $x:succ(k\leq l)$ or $x:succ(n\,dvd\,l)$ and rule \ref{repeat}:\\ 
For a set term $s$ which occurs as $|s|$ in $l$ we introduce a new variable $y$ and add $(x,y):s$. The first two component of $\psi_S(x)$ remains unchanged. Because we can apply this rule we have $n(x,k,S)>n(x,l,S)$ and we have no set term $s$ which occurs in $k$ and in $l$ with the same sign. That means that by adding a new successor to $l$ it can never be a successor to $k$, too. Therefore only $n(x,l,S)$ increases which means $n(x,k,S)\unlhd n(x,l,S)$ decreases and hence also the third component.\\
In $S^\prime$ exists now a new tuple $\psi_{S^\prime}(y)$. But since it was introduced by the assertion $x:succ(c)$, $c\in\{k<l,k\leq l,n\,dvd\,l\}$, the first component of it is always smaller then the first component of $\psi_S(x)$.\\
For any variable $z$ such that $z\neq y$. The tuple $\psi_S(z)$ is unaffected. It can only be affected by the rules if $z$ is a predecessor of $y$. But by Lemma 1.2 that would mean that $z=x$.\\
Altogether $\Psi(S^\prime)$ can be obtained from $\Psi(S)$ by replacing $\psi_S(x)$ with the two smaller tuples $\psi_{S^\prime}(x)$ and $\psi_{S^\prime}(y)$.
\item $S^\prime$ is obtained by the application of rule \ref{exceeded} on $x:succ(k<l)$ or $x:succ(k\leq l)$:\\
The first and second component of $\psi_S(x)$ remain unchanged. The third component also remains unchanged: Because we can apply rule \ref{exceeded} we have $l\in\mathbb{N}$ and therefore $n(x,k,S)>l$ which means the integer in this multiset is $0$. By merging two successor we have $n(x,k,S)\geq n(x,l,S)$ which means the asymmetrical difference $n(x,k,S)\unlhd  n(x,l,S)$ is still $0$. The fourth component can decreases: By merging two successor $y_1, y_2$ the two corresponding entries in the multiset are removed and a new one is added. The new variable $y$ has all assertions of the two successors which means that for some assertions $x:succ(s_1\subseteq s_2)$, such that $(x,y_1):s_1\in S$ and $(x,y_2):s_2\in S$ but $(x,y_1):s_1\not\in S$ and $(x,y_2):s_2\not\in S$, we have after the rule application $(x,y):s_1\in S$ and $(x,y):s_2\in S$ which means that $n_{sc}(x,y)>n_{sc}(x,y_1)$ and $n_{sc}(x,y)>n_{sc}(x,y_2)$. Therefore $n_{sc}(x)-n_{sc}(x,y)$ is smaller than $n_{sc}(x)-n_{sc}(x,y_1)$ or $n_{sc}(x)-n_{sc}(x,y_2)$. The fourth component can not increases because that would mean that by merging two successors we had lost assertions regarding $y_1$ and $y_2$. The fifth component decreases because we have one successor less and therefore $\psi_{S^\prime}(x)$ is smaller than $\psi_S(x)$. The new tuple $\psi_{S^\prime}(y)$ is also smaller than $\psi_S(x)$ because $y$ has the same assertions of the two merged successors whose first component are always smaller than the first component of $\psi_S(x)$ because of Lemma 1.3.\\
No other tuples $\psi_S(z)$ are affected because otherwise $z$ must be a predecessor of $y$ and by Lemma 1.2 $z=x$.\\
Therefore $\Psi(S^\prime)$ can be obtained from $\Psi(S)$ by deleting the tuples of the two merged successors and by replacing $\psi(x)$ with the smaller tuples $\psi_{S^\prime}(x)$ and $\psi_{S^\prime}(y)$.
\item $S^\prime$ is obtained by the application of rule \ref{s} on $x:succ(s_1\subseteq s_2)$ and rule \ref{repeat}:\\ 
After rule \ref{s} $S$ contains $(x,y):s_2$. Then rule \ref{repeat} is applied until inapplicable. After rule \ref{repeat} we can have multiple $(x,y):r$, $r\in\mathbf{R}$, and/or $y:C$. The first and second component do not change. The third component also does not change because we do not add more successors. The fourth component always decreases because the number $n_{sc}(x,y)$ increases. For any $y:C$ $\psi_S(x)$ remains unchanged but we know that the first component of $\psi_S^\prime(y)$ is smaller then the first component of $\psi_S(x)$ by Lemma 1.3.\\
For any variable $z$ such that $z\neq y$ the tuple $\psi_S(z)$ is unaffected. It can only be affected by the rules if $z$ is a predecessor of $y$. But by Lemma 1.2 that would mean that $z=x$.\\
Therefore $\Psi(S^\prime)$ can be obtained from $\Psi(S)$ by deleting $\psi_S(y)$ and by replacing $\psi_S(x)$ with the two smaller septuples $\psi_{S^\prime}(x)$ and $\psi_{S^\prime}(y)$.
\end{enumerate}
\end{proof}
\begin{mylem}
If the Tableau-algorithm terminates without a clash then $S$ is satisfiable
\end{mylem}
\begin{proof}$ $\\
Again the proof is sectioned by the obtained derived sets.\\
Let $\mathcal{I}(S^\prime)$ be the induced interpretation of the ABox $S^\prime$ created by the Tableau-algorithm from $S$. We show that $\pi_{\mathcal{I}(S^\prime)}$ satisfies $S^\prime$.\\
We start with the simple assertions $x:C$ and $(x,y):r$ for $C\in\mathbf{C}$ and $r\in\mathbf{R}$ (induction base): By the definition of induced interpretation we assign $\pi_{\mathcal{I}(S^\prime)}(x):=x^{\mathcal{I}(S^\prime)}\in C^{\mathcal{I}(S^\prime)}$. Also by the definition of induced interpretation for every $(x,y):r\in S^\prime$ we have $(\pi_{\mathcal{I}(S^\prime)}(x),\pi_{\mathcal{I}(S^\prime)}(y)):=(x^{\mathcal{I}(S^\prime)},y^{\mathcal{I}(S^\prime)})\in r^{\mathcal{I}(S^\prime)}$.\\
Let $S$ be an ABox and $\pi_{\mathcal{I}(S)}$ be an assignment which satisfies $S$ (induction hypothesis).\par
\begin{enumerate}
\item If we can apply rule \ref{cap} and obtain $S^\prime$ then there must be an assignment $x:C_1\sqcap C_2\in S$. By the definition of induced interpretation and by the hypothesis we already have $\pi_{\mathcal{I}(S^\prime)}(x)\in C_1^{\mathcal{I}(S^\prime)}$ and $\pi_{\mathcal{I}(S^\prime)}(x)\in C_2^{\mathcal{I}(S^\prime)}$. By adding $x:C_1$ and $x:C_2$ we do not change $\mathcal{I}(S)$. Hence $\mathcal{I}(S^\prime):=\mathcal{I}(S)$ and $\pi_{\mathcal{I}(S^\prime)}:=\pi_{\mathcal{I}(S)}$ satisfies $S^\prime$.
\item If we can apply rule \ref{cup} and obtain $S^\prime$ then there must be an assignment $x:C_1\sqcup C_2\in S$. Like above by the definition of the induced interpretation we have $\pi_{\mathcal{I}(S^\prime)}(x):=x^{\mathcal{I}(S^\prime)}$. We also know that $x^{\mathcal{I}(S^\prime)}$ is in $C_1^{\mathcal{I}(S^\prime)}\cup C_2^{\mathcal{I}(S^\prime)}$. By adding either $x:C_1$ or $x:C_2$ we do not change $\mathcal{I}(S)$. Hence $\mathcal{I}(S^\prime):=\mathcal{I}(S)$ and $\pi_{\mathcal{I}(S^\prime)}:=\pi_{\mathcal{I}(S)}$ satisfies $S^\prime$.
\item If we can apply rule \ref{choose} and obtain $S^\prime$ then we have an assertion $x:succ(k\leq l)$ and a successor $y$ such that $(x,y):k^\prime\in S$, $k^\prime$ occurs in $k$. We then choose between $(x,y):s$ and $(x,y):s^\neg$ for all $|s|$ in $k$ then apply rule \ref{repeat} until this rule is inapplicable. That means at the end we might add several assertions of the form $x:C$ and $(x,y):r$. In case we add $x:C$ we also add $x^{\mathcal{I}(S^\prime)}$ to $C^{\mathcal{I}(S^\prime)}$. Therefore in this case $\pi_{\mathcal{I}(S^\prime)}$ satisfies $S^\prime$. In case we add $(x,y):r$ we also add $(x^{\mathcal{I}(S^\prime)},y^{\mathcal{I}(S^\prime)})$ to $r^{\mathcal{I}(S^\prime)}$. Hence $\pi_{\mathcal{I}(S^\prime)}$ satisfies $S^\prime$.
\item If we can apply rule \ref{chooserole} and obtain $S^\prime$ then we have an assertion $(x,y):k$ but for every role name $r$ we do not have $(x,y):r\in S$, where $r$ has a positive sign in this assertion. After adding $(x,y):r$, $r\in\mathbf{R}$, to $S^\prime$ the element $(x^{\mathcal{I}(S)},y^{\mathcal{I}(S)})$ is also added to $r^{\mathcal{I}(S^\prime)}$. Hence $\pi_{\mathcal{I}(S^\prime)}$ satisfies $S^\prime$.
\item If we can apply rule \ref{leq} and obtain $S^\prime$ then we have an assertion $x:succ(c)$ such that it is violated regarding $x$.\\
We introduce $y$ and add $(x,y):l$ to $S$ and then apply rule \ref{repeat} until this rule is inapplicable. When we introduce $y$ we also add a new element $y^{\mathcal{I}(S^\prime)}$ to $\mathcal{I}(S^\prime)$. For each $y:C$ we add $y^{\mathcal{I}(S^\prime)}$ to $C^{\mathcal{I}(S^\prime)}$ and for each $(x,y):r$, $r\in\mathbf{R}$, we add $(x^{\mathcal{I}(S^\prime)},y^{\mathcal{I}(S^\prime)})$ to $r^{\mathcal{I}(S^\prime)}$. Therefore let $\pi_{\mathcal{I}(S^\prime)}:=\pi_{\mathcal{I}(S)}\cup \{y\mapsto y^{\mathcal{I}(S^\prime)}\}$.
\item If we can apply rule \ref{leq} and obtain $S^\prime$ then we have an assertion $x:succ(k<l)$ or $x:succ(k\leq l)$ such that it is violated regarding $x$. We also have two successors $y_1$ and $y_2$ for which $(x,y):s$ and $(x,y):s$ are in $S$.\\
If we merge both together to $y$ we also have to merge $y_1^{\mathcal{I}(S)}$ and $y_2^{\mathcal{I}(S)}$ to one element $y^{\mathcal{I}(S^\prime)}$. For each $y_i:C\in S$, $i\in\{1,2\}$ we have $y_i^{\mathcal{I}(S)}\in C^{\mathcal{I}(S)}$ and for each $(x,y_i):r$, $r\in\mathbf{R}$ we have $(x^{\mathcal{I}_S},y_i^{\mathcal{I}(S)})\in r^{\mathcal{I}(S)}$ due to the hypothesis. That means that by merging both elements the element $y^{\mathcal{I}(S^\prime)}$ must be in $C^{\mathcal{I}(S^\prime)}$ for every $y_i^{\mathcal{I}(S)}\in C^{\mathcal{I}(S)}$ and the element $(x^{\mathcal{I}(S^\prime)},y^{\mathcal{I}(S^\prime)})$ must be in $r^{\mathcal{I}(S^\prime)}$ for every $(x^{\mathcal{I}(S)},y_i^{\mathcal{I}(S)})\in r^{\mathcal{I}(S)}$. Therefore let $\pi_{\mathcal{I}(S^\prime)}:=\pi_{\mathcal{I}(S)}\backslash\{y_1\mapsto y_1^{\mathcal{I}(S)}, y_2\mapsto y_2^{\mathcal{I}(S)}\}\cup\{y\mapsto y^{\mathcal{I}(S)}\}$ which satisfies $S^\prime$.
\item If we can apply rule \ref{s} and obtain $S^\prime$ then we have an assertion $x:succ(c_1\subseteq c_2)$ and a successor $y$ such that $(x,y):c_1\in S$ but $(x,y):c_2\notin S$. By adding $(x,y):c_2$ to $S$ we have also to add $y:C$ for every concept $C$ in $c_2$ and $(x,y):r$ for every role name $r$ in $c_2$. That means that $x^{\mathcal{I}(S)}$ is added to every $C^{\mathcal{I}(S)}$ and that $(x^{\mathcal{I}(S)},x^{\mathcal{I}(S)})$ is added to every $r^{\mathcal{I}(S)}$. Therefore the assignment $\pi_{\mathcal{I}(S^\prime)}:=\pi_{\mathcal{I}(S)}$ satisfies $S^\prime$.
\end{enumerate}
\end{proof}
\begin{mylem}
If $S$ is satisfiable then the Tableau-algorithm terminates without a clash.
\end{mylem}
\begin{proof}
\end{proof}
\fi
\normalem
\bibliographystyle{abbrv}
\bibliography{ref}
\end{document}
